% $XConsortium: sync.tex /main/8 1996/02/07 17:30:41 swick $
%
% Copyright 1991 by Olivetti Research Limited, Cambridge, England and
% Digital Equipment Corporation, Maynard, Massachusetts.
%
%                        All Rights Reserved
%
% Permission to use, copy, modify, and distribute this software and its 
% documentation for any purpose and without fee is hereby granted, 
% provided that the above copyright notice appear in all copies and that
% both that copyright notice and this permission notice appear in 
% supporting documentation, and that the names of Digital or Olivetti
% not be used in advertising or publicity pertaining to distribution of the
% software without specific, written prior permission.  
%
% DIGITAL AND OLIVETTI DISCLAIM ALL WARRANTIES WITH REGARD TO THIS SOFTWARE,
% INCLUDING ALL IMPLIED WARRANTIES OF MERCHANTABILITY AND FITNESS, IN NO EVENT
% SHALL THEY BE LIABLE FOR ANY SPECIAL, INDIRECT OR CONSEQUENTIAL DAMAGES OR
% ANY DAMAGES WHATSOEVER RESULTING FROM LOSS OF USE, DATA OR PROFITS, WHETHER
% IN AN ACTION OF CONTRACT, NEGLIGENCE OR OTHER TORTIOUS ACTION, ARISING OUT
% OF OR IN CONNECTION WITH THE USE OR PERFORMANCE OF THIS SOFTWARE.

%\documentstyle[a4]{article}
\documentstyle{article}

\setlength{\parindent}{0 pt}
\setlength{\parskip}{6pt}

\newcommand{\system}[1]{{\sc #1}}
\newcommand{\request}[1]{{\bf #1}}
\newcommand{\event}[1]{{\bf #1}}

\newcommand{\param}[1]{{\it #1}}
\newcommand{\error}[1]{{\bf #1}}
\newcommand{\enum}[1]{{\bf #1}}

\newcommand{\eventdef}[1]{\item \event{#1}}
\newcommand{\requestdef}[1]{\item \request{#1}}
\newcommand{\errordef}[1]{\item \error{#1}}
\newcommand{\systemdef}[1]{\item \system{#1}}

\newcommand{\defn}[1]{{\bf #1}}

\newcommand{\tabstopsA}{\hspace*{4cm}\=\hspace*{1cm}\=\hspace*{7cm}\=\kill}
\newcommand{\tabstopsB}{\hspace*{1cm}\=\hspace*{1cm}\=\hspace*{3cm}\=\kill}
\newcommand{\tabstopsC}{\hspace*{1cm}\=\hspace*{1cm}\=\hspace*{5cm}\=\kill}

% commands for formatting the API

\newcommand{\cfunctionname}[1]{\mbox{\tt#1}}
\newcommand{\ctypename}[1]{\mbox{\tt#1}}
\newcommand{\cconst}[1]{\mbox{\tt#1}}
\newcommand{\cargname}[1]{\mbox{\it#1}}
\newcommand{\cstartfunction}[2]{\begin{sloppypar}\begin{samepage}\ctypename{#1}\\ \cfunctionname{#2}\ (}
\newcommand{\cargdecl}[2]{\penalty -1\ctypename{#1} \cargname{#2}}
\newcommand{\cendfunctiondecl}{){\hangafter=2 \hangindent=20pt \raggedright\par}}
\newcommand{\cendfuncdescription}{\end{samepage}\end{sloppypar}}

\newcommand{\cstartmacro}[2]{\begin{sloppypar}\begin{samepage}\ctypename{#1}\\ \cfunctionname{#2}\ (}
\newcommand{\cendmacrodecl}{)\par}
\newcommand{\cendmacrodescription}{\end{samepage}\end{sloppypar}}

% make things easier with all the long names
\spaceskip .3333em plus 5em
\tolerance=2000

\begin{document}

\begin{center}

{\large X SYNCHRONIZATION EXTENSION}\\[10pt]
{\large Version 3.0}\\[15pt]
{\large X Consortium Standard}\\[15pt]
{\large X Version 11, Release 6.1}\\[15pt]
{\it Tim Glauert}\\[0pt]
{\tt thg@cam-orl.co.uk}\\[0pt]
{\bf Olivetti Research / MultiWorks}\\[5pt]
{\it Dave Carver}\\[0pt]
{\tt dcc@athena.mit.edu}\\[0pt]
{\bf Digital Equipment Corporation,}\\[0pt]
{\bf MIT / Project Athena}\\[5pt]
{\it Jim Gettys}\\[0pt]
{\tt jg@crl.dec.com}\\[0pt]
{\bf Digital Equipment Corporation,}\\[0pt]
{\bf Cambridge Research Laboratory}\\[5pt]
{\it David P. Wiggins}\\[0pt]
{\tt dpw@x.org}\\[0pt]
{\bf X Consortium, Inc.}\\[0pt]

\end {center}

Copyright 1991 by Olivetti Research Limited, Cambridge, England and
Digital Equipment Corporation, Maynard, Massachusetts.

{\small Permission to use, copy, modify, and distribute this documentation
for any purpose and without fee is hereby granted, provided that the above
copyright notice appear in all copies. Olivetti, Digital, MIT, and the
X Consortium
make no representations about the suitability for any purpose of the
information in this document. This documentation is provided as is without
express or implied warranty.}

Copyright (c) 1991 X Consortium, Inc.

{\small Permission is hereby granted, free of charge, to any person obtaining a copy
of this software and associated documentation files (the ``Software''), to deal
in the Software without restriction, including without limitation the rights
to use, copy, modify, merge, publish, distribute, sublicense, and/or sell
copies of the Software, and to permit persons to whom the Software is
furnished to do so, subject to the following conditions:

The above copyright notice and this permission notice shall be included in
all copies or substantial portions of the Software.

THE SOFTWARE IS PROVIDED ``AS IS'', WITHOUT WARRANTY OF ANY KIND, EXPRESS OR
IMPLIED, INCLUDING BUT NOT LIMITED TO THE WARRANTIES OF MERCHANTABILITY,
FITNESS FOR A PARTICULAR PURPOSE AND NONINFRINGEMENT.  IN NO EVENT SHALL THE
X CONSORTIUM BE LIABLE FOR ANY CLAIM, DAMAGES OR OTHER LIABILITY, WHETHER IN
AN ACTION OF CONTRACT, TORT OR OTHERWISE, ARISING FROM, OUT OF OR IN
CONNECTION WITH THE SOFTWARE OR THE USE OR OTHER DEALINGS IN THE SOFTWARE.

Except as contained in this notice, the name of the X Consortium shall not be
used in advertising or otherwise to promote the sale, use or other dealings
in this Software without prior written authorization from the X Consortium.}
\eject

\subsection*{Synchronization}

The core X protocol makes no guarantees about the relative order of execution
of requests for different clients. This means that any synchronization between
clients must be done at the client level in an operating-system dependent and
network dependent manner. Even if there was an accepted standard for such
synchronization, the use of a network introduces unpredictable delays between
the synchronization of the clients and the delivery of the resulting requests
to the X server.

The core X protocol also makes no guarantees about the time at which requests
are executed, which means that all clients with real-time constraints must
implement their timing on the host computer. Any such timings are subject to
error introduced by delays within the operating system and network, and are
inefficient due to the need for round-trip requests which keep the client and
server synchronized.

The synchronization extension provides primitives which allow synchronization
between clients to take place entirely within the X server. This removes any
error introduced by the network and makes it possible to synchronize clients
on different hosts running different operating systems. This is important for
multi-media applications where audio, video and graphics data streams are
being synchronized. The extension also provides internal timers within the X
server to which client requests can be synchronized. This allows simple
animation applications to be implemented without any round-trip requests and
makes best use of buffering within the client, network and server.

\subsection*{Description}

The mechanism used by this extension for synchronization within the X server
is to block the processing of requests from a client until a specific
synchronization condition occurs. When the condition occurs the client is
released and processing of requests continues. Multiple clients may block on
the same condition to give inter-client synchronization.  Alternatively, a
single client may block on a condition such as an animation frame-marker.

The extension adds \defn{Counter} and \defn{Alarm} to the set of resources
managed by the server. A counter has a 64-bit integer value which may be
increased or decreased by client requests or by the server internally. A
client can block by sending an \request{Await} request which waits until
one of a set of synchronization conditions, called TRIGGERs, becomes TRUE.

The \request{CreateCounter} request allows a client to create a
\defn{Counter} which can be changed by explicit \request{SetCounter} and
\request{ChangeCounter} requests. These can be used to implement
synchronization between different clients.

There are some counters, called \defn{System Counters}, which are changed by
the server internally rather than by client requests. The effect of any change
to a system counter is not visible until the server has finished processing the
current request. In other words, system counters are apparently updated in the
gaps between the execution of requests rather than during the actual execution
of a request. The extension provides a system counter which advances with the
server time as defined by the core protocol, and it may also provide counters
which advance with the real-world time or which change each time the CRT
screen is refreshed.  Other extensions may provide their own
extension-specific system counters.

The extension provides an \defn{Alarm} mechanism which allows clients to
receive an event on a regular basis when a particular counter is changed.

\subsection*{Types}

Please refer to the X11 Protocol specification as this document uses
syntactic conventions established there and references types defined there.

The following new types are used by the extension.

\begin{tabbing}{l}
SYSTEMCOUNTER: \=\kill
	INT64:	\>64-bit signed integer\\
	COUNTER:\>XID\\
  	VALUETYPE:\>  \{\enum{Absolute},\enum{Relative}\}\\
 	TESTTYPE:\> \{\enum{PositiveTransition},\enum{NegativeTransition},\\
		\>\enum{PositiveComparison},\enum{NegativeComparison}\}\\
	TRIGGER:\>[\\
 		\>\param{counter}:COUNTER,\\
		\>\param{value-type}:VALUETYPE,\\
		\>\param{wait-value}:INT64,\\
		\>\param{test-type}:TESTTYPE\\
		\>]\\
	WAITCONDITION:\>[\\
		\>\param{trigger}:TRIGGER,\\
		\>\param{event-threshold}:INT64\\
		\>]\\
	SYSTEMCOUNTER:\>[\\
		\>\param{name}:STRING8,\\
		\>\param{counter}:COUNTER,\\
		\>\param{resolution}:INT64\\
		\>]\\
	ALARM:	\>XID\\
	ALARMSTATE:\> \{\enum{Active},\enum{Inactive},\enum{Destroyed}\}\\
\end{tabbing}

The COUNTER type defines the client-side handle on a server \defn{Counter}.
The value of a counter is an INT64.

The TRIGGER type defines a test on a counter which is either TRUE or FALSE.
The value of the test is determined by the combination of a test value, the
value of the \param{counter} and the \param{test-type}.

The test value for a trigger is calculated using the \param{value-type} and
\param{wait-value} fields when the trigger is initialized. 
If the \param{value-type} field is not one of the 
named VALUETYPE constants, the request which initialized the trigger
will return a \error{Value} error. If the
\param{value-type} field is \enum{Absolute}, the test value is given by the
\param{wait-value} field. If the \param{value-type} field is
\enum{Relative}, the test value is obtained by adding the
\param{wait-value} field to the value of the counter.  If the
resulting test value would lie outside the range for an INT64, the
request which initialized the trigger will return a
\error{Value} error. If \param{counter} is \enum{None} and the
\param{value-type} is \enum{Relative}, the request which initialized the 
trigger will return a \error{Match} error. 
If \param{counter} is not \enum{None} and does not name a valid
counter, a \error{Counter} error is generated.

If the \param{test-type} is \enum{PositiveTransition}, the trigger is
initialized to FALSE and it will become TRUE when the \param{counter} changes
from a value less than the test value to a value greater than or equal to the
test value. If the \param{test-type} is \enum{NegativeTransition}, the
trigger is initialize to FALSE and it will become TRUE when the
\param{counter} changes from a value greater than the test value to a value
less than or equal to the test value. If the \param{test-type} is
\enum{PositiveComparison} the trigger is TRUE if the counter is greater than or
equal to the test value and FALSE otherwise.  If the \param{test-type} is
\enum{NegativeComparison} the trigger is TRUE if the counter is less than or
equal to the test value and FALSE otherwise. If the \param{test-type}
is not one of the named TESTTYPE constants, the request which
initialized the trigger will return a \error{Value} error.  A trigger
with a \param{counter} value of \enum{None} and a valid \param{test-type}
is always TRUE.

The WAITCONDITION type is simply a trigger with an associated
\param{event-threshold}.  The event threshold is used by the \request{Await}
request to decide whether or not to generate an event to the client after the
trigger has become TRUE. By setting the \param{event-threshold} to an
appropriate value it is possible to detect the situation where an
\request{Await} request was processed {\it after} the TRIGGER became TRUE,
which usually indicates that the server is not processing requests as fast as
the client expects.

The SYSTEMCOUNTER type provides the client with information about a
\defn{System Counter}. The \param{name} field is the textual name of the
counter which identifies the counter to the client. The \param{counter} field
is the client-side handle which should be used in requests which require a
counter. The \param{resolution} field gives the approximate step size of the
system counter. This is a hint to the client that the extension may not be
able to resolve two wait conditions with test values that differ by less than
this step size. A microsecond clock, for example, may advance in steps of 64
microseconds so a counter based on this clock would have a \param{resolution}
of 64.

The only system counter that is guaranteed to be present is called
\system{SERVERTIME}, which counts milliseconds from some arbitrary starting
point. The least-significant 32-bits of this counter track the value of Time
used by the server in Events and Requests. Other system counters may be
provided by different implementations of the extension. The X Consortium will
maintain a registry of system counter names to avoid collisions in the
name space.

An ALARM is the client-side handle on an \defn{Alarm} resource.

\subsection*{Errors}

\begin{description}

\errordef{Counter}

This error is generated if the value for a COUNTER argument in a request does
not name a defined COUNTER.

\errordef{Alarm}

This error is generated if the value for an ALARM argument in a request does
not name a defined ALARM.

\end{description}

\subsection*{Requests}

\begin{description}

\requestdef{Initialize}

\begin{tabular}{l}
	\param{version-major},\param{version-minor}: CARD8
\end{tabular}\\
$\Rightarrow$\\
\begin{tabular}{l}
	\param{version-major},\param{version-minor}: CARD8	
\end{tabular}

This request must be executed before any other requests for this
extension.  If a client violates this rule, the results of all SYNC
requests that it issues are undefined.  The request takes the version
number of the extension which the client wishes to use and returns the
actual version number being implemented by the extension for this
client. The extension may return different version numbers to a client
depending of the version number supplied by that client. This request
should only be executed once for each client connection.

Given two different versions of the SYNC protocol, v1 and v2, v1 is
compatible with v2 if and only if $v1.version\_major = v2.version\_major$
and $v1.version\_minor \leq v2.version\_minor$.  Compatible means that the
functionality is fully supported in an identical fashion in the two
versions.

This document describes major version 3, minor version 0 of the SYNC
protocol.

\requestdef{ListSystemCounters}

$\Rightarrow$\\
\begin{tabular}{l}
	\param{system-counters}: LISTofSYSTEMCOUNTER\\[5pt]
	Errors: \error{Alloc}
\end{tabular}

This request returns a list of all the system counters which are available at
the time the request is executed, which includes the system counters which are
maintained by other extensions. The list returned by this request may change
as counters are created and destroyed by other extensions.

\requestdef{CreateCounter}

\begin{tabular}{l}
	\param{id}: COUNTER\\
 	\param{initial-value}: INT64\\[5pt]
	Errors: \error{IDChoice}, \error{Alloc}
\end{tabular}

This request creates a counter and assigns the identifier \param{id} to it.
The counter value is initialized to \param{initial-value} and there are no
clients waiting on the counter.

\requestdef{DestroyCounter}

\begin{tabular}{l}
	\param{counter}: COUNTER\\[5pt]
	Errors: \error{Counter},\error{Access}
\end{tabular}

This request destroys the given counter and sets the \param{counter} fields
for all triggers which specify this counter to \enum{None}. All clients
waiting on the counter are released and a \event{CounterNotify} event with the
\param{destroyed} field set to TRUE is sent to each waiting client,
regardless of the \param{event-threshold}.  All alarms specifying the counter
become \enum{Inactive} and an \event{AlarmNotify} event with a \param{state}
field of \enum{Inactive} is generated. A counter is destroyed automatically
when the connection to the creating client is closed down if the close-down
mode is {\bf Destroy}. An \error{Access} error is generated if \param{counter}
is a system counter. A \error{Counter} error is generated if \param{counter}
does not name a valid counter.

\requestdef{QueryCounter}

\begin{tabular}{l}
	\param{counter}: COUNTER\\
\end{tabular}\\
$\Rightarrow$\\
\begin{tabular}{l}
	\param{value}: INT64\\[5pt]
	Errors: \error{Counter}
\end{tabular}

This request returns the current value of the given counter or a generates
\error{Counter} error if \param{counter} does not name a valid counter.

\requestdef{Await}

\begin{tabular}{l}
	\param{wait-list}: LISTofWAITCONDITION\\[5pt]
	Errors: \error{Counter}, \error{Alloc}, \error{Value}
\end{tabular}

When this request is executed, the triggers in the \param{wait-list} are
initialized using the \param{wait-value} and \param{value-type} fields, as
described in the definition of TRIGGER above. The processing of further
requests for the client is blocked until one or more of the triggers becomes
TRUE. This may happen immediately, as a result of the initialization, or at
some later time, as a result of a subsequent \request{SetCounter},
\request{ChangeCounter} or \request{DestroyCounter} request.

A \error{Value} error is generated if \param{wait-list} is empty.

When the client becomes unblocked, each trigger is checked to determine
whether a \event{CounterNotify} event should be generated. The difference
between the \param{counter} and the test value is calculated by
subtracting the test value from the value of the \param{counter}. If the
\param{test-type} is \enum{PositiveTransition} or \enum{PositiveComparison}, a \event{CounterNotify} event is generated if the
difference is at least \param{event-threshold}. If the \param{test-type} is
\enum{NegativeTransition} or \enum{NegativeComparison}, a
\event{CounterNotify} event is generated if the difference is at most
\param{event-threshold}. If the difference lies outside the range for an
INT64, an event is not generated.

This threshold check is made for each trigger in the list and a
\event{CounterNotify} event is generated for every trigger for which
the check succeeds. The check for \enum{CounterNotify} events is performed
even if one of the triggers is TRUE when the request is first executed. Note
that a \event{CounterNotify} event may be generated for a \param{trigger} that
is FALSE if there are multiple triggers in the request. A
\event{CounterNotify} event with the \param{destroyed} flag set to TRUE is
always generated if the counter for one of the triggers is destroyed.

\requestdef{ChangeCounter}

\begin{tabular}{l}
	\param{counter}: COUNTER\\
	\param{amount}: INT64\\[5pt]
	Errors: \error{Counter},\error{Access},\error{Value}
\end{tabular}

This request changes the given counter by adding \param{amount} to the current
counter value. If the change to this counter satisfies a trigger for which a
client is waiting, that client is unblocked and one or more
\event{CounterNotify} events may be generated. If the change to the counter
satisfies the trigger for an alarm, an \event{AlarmNotify} event is generated
and the alarm is updated.  An \error{Access} error is generated if
\param{counter} is a system counter. A \error{Counter} error is generated if
\param{counter} does not name a valid counter. If the resulting value for the
counter would be outside the range for an INT64, a \error{Value} error is
generated and the counter is not changed.

It should be noted that {\it all} the clients whose triggers are satisfied by
this change are unblocked, so this request cannot be used to implement mutual
exclusion.

\requestdef{SetCounter}

\begin{tabular}{l}
	\param{counter}: COUNTER\\
	\param{value}: INT64\\[5pt]
	Errors: \error{Counter},\error{Access}
\end{tabular}

This request sets the value of the given counter to \param{value}. The effect
is equivalent to executing the appropriate \request{ChangeCounter} request to
change the counter value to \param{value}. An \error{Access} error is
generated if \param{counter} names a system counter. A \error{Counter} error
is generated if \param{counter} does not name a valid counter.

\requestdef{CreateAlarm}

\begin{tabular}{l}
	\param{id}: ALARM\\
 	\param{values-mask}: CARD32\\
        \param{values-list}: LISTofVALUE\\[5pt]
 	Errors: \error{IDChoice},\error{Counter},\error{Match},\error{Value},\error{Alloc}
\end{tabular}

This request creates an alarm and assigns the identifier \param{id} to it. The
\param{values-mask} and \param{values-list} specify the attributes which are
to be explicitly initialized. The attributes for an Alarm and their defaults
are:

\begin{center}
\begin{tabular}{l|l|ll}
Attribute	& Type		& Default \\
\hline
trigger		& TRIGGER	& counter	& \enum{None}\\
		&		& value-type	& \enum{Absolute}\\
		&		& value	& 0\\
		&		& test-type	& \enum{PositiveComparison}\\
delta		& INT64		& 1 \\
events		& BOOL		& TRUE
\end{tabular}
\end{center}

The \param{trigger} is initialized as described in the definition of TRIGGER,
with an error being generated if necessary.

If the \param{counter} is \enum{None}, the state of the alarm is set to
\enum{Inactive}, else it is set to \enum{Active}.

Whenever the \param{trigger} becomes TRUE, either as a result of this request
or as the result of a \request{SetCounter}, \request{ChangeCounter},
\request{DestroyCounter} or \request{ChangeAlarm} request, an
\event{AlarmNotify} event is generated and the alarm is updated. The alarm is
updated by repeatedly adding \param{delta} to the \param{value} of the
\param{trigger} and re-initializing it until it becomes FALSE. If this update
would cause \param{value} to fall outside the range for an INT64, or if the
\param{counter} value is \enum{None}, or if the
\param{delta} is 0 and \param{test-type} is \enum{PositiveComparison} or
\enum{NegativeComparison}, no change is made to \param{value} and the alarm
state is changed to \enum{Inactive} before the event is generated. No further
events are generated by an \enum{Inactive} alarm until a \request{ChangeAlarm}
or \request{DestroyAlarm} request is executed.

If the \param{test-type} is \enum{PositiveComparison} or
\enum{PositiveTransition} and \param{delta} is less than zero, or
if the \param{test-type} is \enum{NegativeComparison} or
\enum{NegativeTransition} and \param{delta} is greater than zero,
a \error{Match} error is generated.

The \param{events} value enables or disables delivery of \event{AlarmNotify}
events to the requesting client.  The alarm keeps a separate event flag for
each client so that other clients may select to receive events from this
alarm.

An \event{AlarmNotify} event is always generated at some time after the
execution of a \request{CreateAlarm} request. This will happen immediately if
the \param{trigger} is TRUE or it will happen later when the
\param{trigger} becomes TRUE or the Alarm is destroyed.

\requestdef{ChangeAlarm}

\begin{tabular}{l}
	\param{id}: ALARM\\
	\param{values-mask}: CARD32\\
	\param{values-list}: LISTofVALUE\\[5pt]
	Errors: \error{Alarm},\error{Counter},\error{Value},\error{Match}
\end{tabular}

This request changes the parameters of an Alarm. All of the parameters
specified for the \request{CreateAlarm} request may be changed using this
request. The \param{trigger} is re-initialized and an \event{AlarmNotify}
event is generated if appropriate, as explained in the description of the
\request{CreateAlarm} request.

Changes to the \param{events} flag affect the event delivery to the requesting
client only, and may be used by a client to select or de-select event delivery
from an alarm created by another client.

The order in which attributes are verified and altered is
server-dependent.  If an error is generated, a subset of the
attributes may have been altered.

\requestdef{DestroyAlarm}

\begin{tabular}{l}
	\param{alarm}: ALARM\\[5pt] 	Errors: \error{Alarm}
\end{tabular}

This request destroys an alarm. An alarm is automatically destroyed
when the creating client is closed down if the close-down mode is {\bf
Destroy}. When an alarm is destroyed, an \event{AlarmNotify} event is
generated with a \param{state} value of \enum{Destroyed}.

\requestdef{QueryAlarm}

\begin{tabular}{l}
	\param{alarm}: ALARM\\
\end{tabular}\\
$\Rightarrow$\\
\begin{tabular}{l}
	\param{trigger}: TRIGGER\\
	\param{delta}: INT64\\
	\param{events}: ALARMEVENTMASK\\
	\param{state}: ALARMSTATE\\[5pt]
	Errors: \error{Alarm}
\end{tabular}

This request retrieves the current parameters for an Alarm.

\requestdef{SetPriority}

\begin{tabular}{l}
	\param{client-resource}: XID\\
	\param{priority}: INT32\\[5pt]
	Errors: \error{Match}
\end{tabular}

This request changes the scheduling priority of the client which created
\param{client-resource}. If \param{client-resource} is \enum{None} then the
priority for the client making the request is changed. A \error{Match} error
is generated if \param{client-resource} is not \enum{None} and does not name
an existing resource in the server.  For any two \param{priority} values,
{\tt A} and {\tt B}, {\tt A} is higher priority if and only if {\tt A} is
greater than {\tt B}.

The priority of a client is set to 0 when the initial client connection is
made.

The effect of different client priorities depends on the particular
implementation of the extension, and in some cases it may have no effect at
all. However, the intention is that higher priority clients will have their
requests executed before those of lower priority clients.

For most animation applications, it is desireable that animation clients are
given priority over non-real-time clients. This improves the smoothness of the
animation on a loaded server. Because a server is free to implement very strict
priorities, processing requests for the highest priority client to the
exclusion of all others, it is important that a client which may potentially
monopolize the whole server, such as an animation which produces continuous
output as fast as it can with no rate control, is run at low rather than high
priority.

\requestdef{GetPriority}

\begin{tabular}{l}
	\param{client-resource}: XID\\
\end{tabular}\\
$\Rightarrow$\\
\begin{tabular}{l}
	\param{priority}: INT32\\[5pt]
	Errors: \error{Match}
\end{tabular}

This request returns the scheduling priority of the client which created
\param{client-resource}. If \param{client-resource} is \enum{None} then the
priority for the client making the request is returned. A \error{Match} error
is generated if \param{client-resource} is not \enum{None} and does not name
an existing resource in the server.

\end{description}

\subsection*{Events}

\begin{description}

\eventdef{CounterNotify}

\begin{tabular}{l}
	\param{counter}: COUNTER \\
 	\param{wait-value}: INT64 \\
	\param{counter-value}: INT64 \\
 	\param{time}: TIME \\
	\param{count}: CARD16 \\
	\param{destroyed}: BOOL
\end{tabular}

\event{CounterNotify} events may be generated when a client becomes unblocked
after an \request{Await} request has been processed.
\param{wait-value} is the value being waited for and
\param{counter-value} is the actual value of the counter at the time
the event was generated. The
\param{destroyed} flag is TRUE if this request was generated as the
result of the destruction of the counter, and FALSE otherwise.
\param{time} is the server time at which the event was generated.

When a client is unblocked, all the \event{CounterNotify} events for the
\request{Await} request are generated contiguously. If
\param{count} is 0 there are no more events to follow for this request. If
\param{count} is $n$, there are at least $n$ more events to follow.

\eventdef{AlarmNotify}

\begin{tabular}{l}
	\param{alarm}: ALARM \\
 	\param{counter-value}: INT64 \\
	\param{alarm-value}: INT64 \\
 	\param{state}: ALARMSTATE \\
 	\param{time}: TIME
\end{tabular}

An \event{AlarmNotify} event is generated when an alarm is triggered.
\param{alarm-value} is the test value of the trigger in the alarm when it was
triggered, \param{counter-value} is the value of the counter which triggered
the alarm and \param{time} is the server time at which the event was
generated. \param{state} is the new state of the alarm. If \param{state} is
\enum{Inactive}, no more events will be generated by this alarm until a
\request{ChangeAlarm} request is executed, the alarm is destroyed or the
counter for the alarm is destroyed.

\end{description}

\section*{Encoding}

Please refer to the X11 Protocol Encoding document as this section uses
syntactic conventions established there and references types defined there.

The name of this extension is ``SYNC''.

\subsection*{New Types}

The following new types are used by the extension.

\begin{tabbing}
\tabstopsC
ALARM: CARD32\\
ALARMSTATE:\\
\tabstopsB
	\> 0	\> Active \\
	\> 1	\> Inactive \\
	\> 2	\> Destroyed\\
\tabstopsC
COUNTER: CARD32\\
INT64: 64-bit signed integer\\
SYSTEMCOUNTER:\\
	\> 4	\> COUNTER	\> counter \\
	\> 8	\> INT64	\> resolution\\
	\> 2	\> n		\> length of name in bytes\\
	\> n	\> STRING8	\> name \\
	\> p	\> 		\> pad,p=pad(n+2)\\
TESTTYPE:\\
\tabstopsB
	\> 0	\> PositiveTransition \\
	\> 1	\> NegativeTransition \\
	\> 2	\> PositiveComparison \\
	\> 3	\> NegativeComparison \\
\tabstopsC
TRIGGER:\\
	\> 4	\> COUNTER	\> counter \\
	\> 4	\> VALUETYPE	\> wait-type \\	
	\> 8	\> INT64	\> wait-value \\
	\> 4	\> TESTTYPE	\> test-type \\
VALUETYPE:\\
\tabstopsB
	\> 0	\> Absolute \\
	\> 1	\> Relative \\
\tabstopsC
WAITCONDITION:\\
	\> 20	\> TRIGGER	\> trigger \\
	\> 8	\> INT64	\> event threshold\\
\end{tabbing}

An INT64 is encoded in 8 bytes with the most significant 4 bytes
first followed by the least significant 4 bytes.  Within these
4 byte groups, the byte ordering determined during connection setup
is used.

\subsection*{Errors}

\begin{tabbing}
\tabstopsC
{\bf Counter}\\
	\> 1	\> 0		\> Error \\
	\> 1	\> Base + 0	\> code \\
	\> 2	\> CARD16	\> sequence number \\
	\> 4	\> CARD32	\> bad counter \\
	\> 2	\> CARD16	\> minor opcode \\
	\> 1	\> CARD8	\> major opcode \\
	\> 21	\>		\> unused \\
{\bf Alarm}\\
	\> 1	\> 0		\> Error \\
	\> 1	\> Base + 1	\> code \\
	\> 2	\> CARD16	\> sequence number \\
	\> 4	\> CARD32	\> bad alarm \\
	\> 2	\> CARD16	\> minor opcode \\
	\> 1	\> CARD8	\> major opcode \\
	\> 21	\>		\> unused \\
\end{tabbing}

\subsection*{Requests}

\renewcommand{\thefootnote}{\fnsymbol{footnote}}
\setcounter{footnote}{1}
\setlength{\topsep}{0pt}	%vertical space before and after tabbing
\begin{tabbing}
\tabstopsC
{\bf Initialize}\\
	\> 1	\> CARD8	\> major opcode \\
	\> 1	\> 0		\> minor opcode \\
	\> 2	\> 2		\> request length \\
	\> 1	\> CARD8	\> major version \\
	\> 1	\> CARD8	\> minor version \\
	\> 2	\> 		\> unused \\
$\Rightarrow$\\
	\> 1	\> 1		\> Reply \\
	\> 1	\>		\> unused \\
	\> 2	\> CARD16	\> sequence number \\
	\> 4	\> 0		\> reply length \\
	\> 1	\> CARD8	\> major version \\
	\> 1	\> CARD8	\> minor version \\
	\> 2	\>		\> unused \\
	\> 20	\>		\> unused \\
\\
{\bf ListSystemCounters}\\
	\> 1	\> CARD8	\> major opcode \\
	\> 1	\> 1		\> minor opcode \\
	\> 2	\> 1		\> request length \\
$\Rightarrow$\\
	\> 1	\> 1		\> Reply \\
	\> 1	\>		\> unused \\
	\> 2	\> CARD16	\> sequence number \\
	\> 4	\> {\it variable} \> reply length \\
	\> 4	\> INT32	\> list length \\
	\> 20	\>		\> unused \\
	\> 4n   \> list of SYSTEMCOUNTER \> system counters \\
\\
{\bf CreateCounter}\\
	\> 1	\> CARD8	\> major opcode \\
	\> 1	\> 2		\> minor opcode \\
	\> 2	\> 4		\> request length \\
	\> 4	\> COUNTER	\> id\\
	\> 8	\> INT64	\> initial value\\
\\
{\bf DestroyCounter}\\
	\> 1	\> CARD8	\> major opcode \\
	\> 1	\> 6		\> minor opcode\footnotemark[1] \\
	\> 2	\> 2		\> request length \\
	\> 4	\> COUNTER	\> counter
\end{tabbing}
\footnotetext{A previous version of this document gave an incorrect
minor opcode.}
\setlength{\topsep}{0pt}	%vertical space before and after tabbing
\begin{tabbing}
\tabstopsC
{\bf QueryCounter}\\
	\> 1	\> CARD8	\> major opcode \\
	\> 1	\> 5		\> minor opcode\footnotemark[1] \\
	\> 2	\> 2		\> request length \\
	\> 4	\> COUNTER	\> counter \\
$\Rightarrow$\\
	\> 1	\> 1		\> Reply \\
	\> 1	\>		\> unused \\
	\> 2	\> CARD16	\> sequence number \\
	\> 4	\> 0		\> reply length \\
	\> 8	\> INT64	\> counter value \\
	\> 16	\>		\> unused\\
\\
{\bf Await}\\
	\> 1	\> CARD8	\> major opcode \\
	\> 1	\> 7		\> minor opcode\footnotemark[1] \\
	\> 2	\> 1 + 7*n	\> request length \\
	\> 28n	\> LISTofWAITCONDITION \> wait conditions
\end{tabbing}
\footnotetext{A previous version of this document gave an incorrect
minor opcode.}
\setlength{\topsep}{0pt}	%vertical space before and after tabbing
\begin{tabbing}
\tabstopsC
{\bf ChangeCounter}\\
	\> 1	\> CARD8	\> major opcode \\
	\> 1	\> 4		\> minor opcode\footnotemark[1] \\
	\> 2	\> 4		\> request length \\
	\> 4	\> COUNTER	\> counter \\
	\> 8	\> INT64	\> amount \\
\\
{\bf SetCounter}\\
	\> 1	\> CARD8	\> major opcode \\
	\> 1	\> 3		\> minor opcode\footnotemark[1] \\
	\> 2	\> 4		\> request length \\
	\> 4	\> COUNTER	\> counter \\
	\> 8	\> INT64	\> value \\
\\
{\bf CreateAlarm}\\
	\> 1	\> CARD8	\> major opcode \\
	\> 1	\> 8		\> minor opcode \\
	\> 2	\> 3+n		\> request length \\
	\> 4	\> ALARM	\> id \\
	\> 4	\> BITMASK	\> values mask\\
\tabstopsB
	\>	\> \#x00000001	\> counter \\
	\>	\> \#x00000002	\> value-type \\
	\>	\> \#x00000004	\> value \\
	\>	\> \#x00000008	\> test-type \\
	\>	\> \#x00000010	\> delta \\
	\>	\>  \#x00000020	\> events \\
\tabstopsC
	\> 4n	\> LISTofVALUE	\> values\\
\tabstopsB
VALUES\\
	\> 4	\> COUNTER	\> counter\\
	\> 4	\> VALUETYPE	\> value-type \\
	\> 8	\> INT64	\> value \\
	\> 4	\> TESTTYPE	\> test-type \\
	\> 8	\> INT64	\> delta \\
	\> 4	\> BOOL		\> events\\
\tabstopsC
\\
{\bf ChangeAlarm}\\
	\> 1	\> CARD8	\> major opcode \\
	\> 1	\> 9		\> minor opcode \\
	\> 2	\> 3+n		\> request length \\
	\> 4	\> ALARM	\> id \\
	\> 4	\> BITMASK	\> values mask \\
	\> 	\> encodings as for {\bf CreateAlarm}\\
	\> 4n	\> LISTofVALUE	\> values\\
	\>	\> encodings as for {\bf CreateAlarm}\\
\\
{\bf DestroyAlarm}\\
	\> 1	\> CARD8	\> major opcode \\
	\> 1	\> 11		\> minor opcode\footnotemark[1] \\
	\> 2	\> 2		\> request length \\
	\> 4	\> ALARM	\> alarm
\end{tabbing}
\footnotetext{A previous version of this document gave an incorrect
minor opcode.}
\setlength{\topsep}{0pt}	%vertical space before and after tabbing
\begin{tabbing}
\tabstopsC
{\bf QueryAlarm}\\
	\> 1	\> CARD8	\> major opcode \\
	\> 1	\> 10		\> minor opcode\footnotemark[1] \\
	\> 2	\> 2		\> request length \\
	\> 4	\> ALARM	\> alarm \\
$\Rightarrow$\\
	\> 1	\> 1		\> Reply \\
	\> 1	\>		\> unused \\
	\> 2	\> CARD16	\> sequence number \\
	\> 4	\> 2		\> reply length \\
	\> 20	\> TRIGGER	\> trigger \\
	\> 8	\> INT64	\> delta \\
	\> 1	\> BOOL		\> events \\
	\> 1	\> ALARMSTATE	\> state \\
	\> 2	\>		\> unused \\
\\
{\bf SetPriority}\\
	\> 1	\> CARD8	\> major opcode \\
	\> 1	\> 12		\> minor opcode \\
	\> 2	\> 3		\> request length \\
	\> 4	\> CARD32	\> id \\
	\> 4	\> INT32	\> priority \\
\\
{\bf GetPriority}\\
	\> 1	\> CARD8	\> major opcode \\
	\> 1	\> 13		\> minor opcode \\
	\> 2	\> 1		\> request length \\
	\> 4	\> CARD32	\> id \\
$\Rightarrow$\\
	\> 1	\> 1		\> Reply \\
	\> 1	\>		\> unused \\
	\> 2	\> CARD16	\> sequence number \\
	\> 4	\> 0		\> reply length \\
	\> 4	\> INT32	\> priority \\
	\> 20	\>		\> unused\\
\end{tabbing}

\subsection*{Events}

\begin{tabbing}
\tabstopsC
{\bf CounterNotify}\\
	\> 1	\> Base + 0	\> code \\
	\> 1	\> 0		\> kind \\
	\> 2	\> CARD16	\> sequence number \\
	\> 4	\> COUNTER	\> counter \\
	\> 8	\> INT64	\> wait value \\
	\> 8	\> INT64	\> counter value \\
	\> 4	\> TIME		\> timestamp \\
	\> 2	\> CARD16	\> count \\
	\> 1	\> BOOL		\> destroyed \\
	\> 1	\> 		\> unused \\
\\
{\bf AlarmNotify}\\
	\> 1	\> Base + 1	\> code \\
	\> 1	\> 1		\> kind \\
	\> 2	\> CARD16	\> sequence number \\
	\> 4	\> ALARM	\> alarm \\
	\> 8	\> INT64	\> counter value \\
	\> 8	\> INT64	\> alarm value \\
	\> 4	\> TIME		\> timestamp \\
	\> 1	\> ALARMSTATE	\> state \\
	\> 3	\>		\> unused\\
\end{tabbing}

\section*{C Language Binding}

The C routines provide direct access to	the protocol and add
no additional semantics.

The include file for this extension is \verb|<X11/extensions/sync.h>|.

Most of the names in the language binding are derived from the
protocol names by prepending XSync to the protocol name and changing
the capitalization.

\subsection*{C Functions}

Most of the following functions generate SYNC protocol requests.

\cstartfunction{Status}{XSyncQueryExtension}
\cargdecl{Display *}{dpy},
\cargdecl{int *}{event\_base\_return},
\cargdecl{int *}{error\_base\_return}
\cendfunctiondecl

If \cargname{dpy} supports the SYNC extension, this function returns
\cconst{True}, sets *\cargname{event\_base\_return} to the event number for the
first SYNC event, and sets
*\cargname{error\_base\_return} to the error number for the first SYNC
error.  If \cargname{dpy} does not support the SYNC extension, this function
returns \cconst{False}.
\cendfuncdescription

\cstartfunction{Status}{XSyncInitialize}
\cargdecl{Display *}{dpy},
\cargdecl{int *}{major\_version\_return},
\cargdecl{int *}{minor\_version\_return}
\cendfunctiondecl

Sets *\cargname{major\_version\_return} and
*\cargname{minor\_version\_return} to the major/minor SYNC protocol
version supported by the server.  If the XSync library is compatible
with the version returned by the server, this function returns \cconst{True}.
If \cargname{dpy} does not support the SYNC extension, or if there was an error during
communication with the server, or if the server and library protocol
versions are incompatible, this function returns \cconst{False}.  The only
XSync function that may be called before this function is
\cfunctionname{XSyncQueryExtension}.  If a client violates this rule,
the effects of all XSync calls that it makes are undefined.
\cendfuncdescription

\cstartfunction{XSyncSystemCounter *}{XSyncListSystemCounters}
\cargdecl{Display *}{dpy},
\cargdecl{int *}{n\_counters\_return}
\cendfunctiondecl

Returns a pointer to an array of system counters supported by the
display, and sets *\cargname{n\_counters\_return} to the number of
counters in the array.  The array should be freed with
\cfunctionname{XSyncFreeSystemCounterList}.  If \cargname{dpy} does not
support the SYNC extension, or if there was an error during communication with the
server, or if the server does not support any system counters, this
function returns \cconst{NULL}.

\ctypename{XSyncSystemCounter} has the following fields:

\begin{tabular}{lll}
\ctypename{char *} & name; & /* null-terminated name of system counter */\\
\ctypename{XSyncCounter} & counter; & /* counter id of this system counter */\\
\ctypename{XSyncValue} & resolution; & /* resolution of this system counter */\\
\end{tabular}
\cendfuncdescription

\cstartfunction{void}{XSyncFreeSystemCounterList}
\cargdecl{XSyncSystemCounter *}{list}
\cendfunctiondecl

Frees the memory associated with the system counter list returned by
\cfunctionname{XSyncListSystemCounters}.
\cendfuncdescription

\cstartfunction{XSyncCounter}{XSyncCreateCounter}
\cargdecl{Display *}{dpy}, 
\cargdecl{XSyncValue}{initial\_value}
\cendfunctiondecl

Creates a counter on the \cargname{dpy} with the given \cargname{initial\_value}
and returns the counter ID.  Returns \cconst{None} if \cargname{dpy} does not
support the SYNC extension.
\cendfuncdescription

\cstartfunction{Status}{XSyncSetCounter}
\cargdecl{Display *}{dpy},
\cargdecl{XSyncCounter}{counter},
\cargdecl{XSyncValue}{value}
\cendfunctiondecl

Sets \cargname{counter} to \cargname{value}.  Returns \cconst{False}
if \cargname{dpy} does not support the SYNC extension, else returns \cconst{True}.
\cendfuncdescription

\cstartfunction{Status}{XSyncChangeCounter}
\cargdecl{Display *}{dpy},
\cargdecl{XSyncCounter}{counter},
\cargdecl{XSyncValue}{value}
\cendfunctiondecl

Adds \cargname{value} to \cargname{counter}.  Returns \cconst{False}
if \cargname{dpy} does not support the SYNC extension, else returns \cconst{True}.
\cendfuncdescription

\cstartfunction{Status}{XSyncDestroyCounter}
\cargdecl{Display *}{dpy}, 
\cargdecl{XSyncCounter}{counter}
\cendfunctiondecl

Destroys \cargname{counter}.  Returns \cconst{False} if \cargname{dpy} does not
support the SYNC extension, else returns \cconst{True}.
\cendfuncdescription

\cstartfunction{Status}{XSyncQueryCounter}
\cargdecl{Display *}{dpy},
\cargdecl{XSyncCounter}{counter},
\cargdecl{XSyncValue *}{value\_return}
\cendfunctiondecl

Sets \cargname{*value\_return} to the current value of
\cargname{counter}.  Returns \cconst{False} if there was an error during
communication with the server, or if \cargname{dpy} does not support
the SYNC extension, else returns \cconst{True}.
\cendfuncdescription

\cstartfunction{Status}{XSyncAwait}
\cargdecl{Display *}{dpy},
\cargdecl{XSyncWaitCondition *}{wait\_list},
\cargdecl{int}{n\_conditions}
\cendfunctiondecl

Awaits on the conditions in \cargname{wait\_list}.
\cargname{n\_conditions} is the number of wait conditions in
\cargname{wait\_list}.  Returns \cconst{False} if \cargname{dpy} does not
support the SYNC extension, else returns \cconst{True}.  The await is processed
asynchronously by the server; this function always returns immediatley
after issuing the request.

\ctypename{XSyncWaitCondition} has the following fields:

\begin{tabular}{lll}
\ctypename{XSyncCounter} & trigger.counter; & /* counter to trigger on */ \\
\ctypename{XSyncValueType} & trigger.value\_type; & /* absolute/relative */ \\
\ctypename{XSyncValue} & trigger.wait\_value; & /* value to compare counter to */ \\
\ctypename{XSyncTestType} & trigger.test\_type;	& /* pos/neg comparison/transtion */ \\
\ctypename{XSyncValue} & event\_threshold; & /* send event if past threshold */ \\
\end{tabular}

\ctypename{XSyncValueType} can be either \cconst{XSyncAbsolute} or \cconst{XSyncRelative}.

\ctypename{XSyncTestType} can be one of \cconst{XSyncPositiveTransition},
\cconst{XSyncNegativeTransition}, \cconst{XSyncPositiveComparison}, or
\cconst{XSyncNegativeComparison}.
\cendfuncdescription

\cstartfunction{XSyncAlarm}{XSyncCreateAlarm}
\cargdecl{Display *}{dpy},
\cargdecl{unsigned long}{values\_mask},
\cargdecl{XSyncAlarmAttributes *}{values}
\cendfunctiondecl

Creates an alarm and returns the alarm ID.  Returns \cconst{None} if the
display does not support the SYNC extension.  \cargname{values\_mask} and \cargname{values}
specify the alarm attributes.

\ctypename{XSyncAlarmAttributes} has the following fields.  The attribute\_mask
column specifies the symbol that the caller should OR into
\cargname{values\_mask} to indicate that the value for the corresponding
attribute was actually supplied.  Default values are used for all
attributes that do not have their attribute\_mask OR'ed into 
\cargname{values\_mask}.
See the protocol
description for CreateAlarm for the defaults.

\begin{tabular}{lll}
type & field name & attribute\_mask \\
\ctypename{XSyncCounter} & trigger.counter; & \cconst{XSyncCACounter} \\
\ctypename{XSyncValueType}& trigger.value\_type; & \cconst{XSyncCAValueType} \\
\ctypename{XSyncValue} & trigger.wait\_value; & \cconst{XSyncCAValue} \\
\ctypename{XSyncTestType} & trigger.test\_type; & \cconst{XSyncCATestType} \\
\ctypename{XSyncValue} & delta;	& \cconst{XSyncCADelta} \\
\ctypename{Bool} & events; & \cconst{XSyncCAEvents} \\
\ctypename{XSyncAlarmState} & state; & client cannot set this \\
\end{tabular}
\cendfuncdescription

\cstartfunction{Status}{XSyncDestroyAlarm}
\cargdecl{Display *}{dpy},
\cargdecl{XSyncAlarm}{alarm}
\cendfunctiondecl

Destroys \cargname{alarm}.  Returns \cconst{False} if \cargname{dpy} does not
support the SYNC extension, else returns \cconst{True}.
\cendfuncdescription

\cstartfunction{Status}{XSyncQueryAlarm}
\cargdecl{Display *}{dpy},
\cargdecl{XSyncAlarm}{alarm},
\cargdecl{XSyncAlarmAttributes *}{values\_return}
\cendfunctiondecl

Sets *\cargname{values\_return} to the alarm's attributes.  Returns
\cconst{False} if there was an error during communication with the
server, or if \cargname{dpy} does not support the SYNC extension, else returns
\cconst{True}.
\cendfuncdescription

\cstartfunction{Status}{XSyncChangeAlarm}
\cargdecl{Display *}{dpy},
\cargdecl{XSyncAlarm}{alarm},
\cargdecl{unsigned long}{values\_mask},
\cargdecl{XSyncAlarmAttributes *}{values}
\cendfunctiondecl

Changes \cargname{alarm}'s attributes.  The attributes to change are specified
as in \cfunctionname{XSyncCreateAlarm}.  Returns \cconst{False} if
\cargname{dpy} does not support the SYNC extension, else returns \cconst{True}.
\cendfuncdescription

\cstartfunction{Status}{XSyncSetPriority}
\cargdecl{Display *}{dpy},
\cargdecl{XID}{client\_resource\_id},
\cargdecl{int}{priority}
\cendfunctiondecl

Sets the priority of the client owning \cargname{client\_resource\_id} to
\cargname{priority}.  If \cargname{client\_resource\_id} is \cconst{None},
sets the caller's priority.  Returns \cconst{False} if \cargname{dpy}
does not support the SYNC extension, else returns \cconst{True}.
\cendfuncdescription

\cstartfunction{Status}{XSyncGetPriority}
\cargdecl{Display *}{dpy},
\cargdecl{XID}{client\_resource\_id},
\cargdecl{int *}{return\_priority}
\cendfunctiondecl

Sets *\cargname{return\_priority} to the priority of the client owning
\cargname{client\_resource\_id}.  If \cargname{client\_resource\_id} is
\cconst{None}, sets *\cargname{return\_priority} to the caller's priority.
Returns \cconst{False} if there was an error
during communication with the server, or if \cargname{dpy} does not
support the SYNC extension, else returns \cconst{True}.
\cendfuncdescription

\subsection*{C Macros/Functions}

The following procedures manipulate 64 bit values.  They are defined
both as macros and as functions.  By default, the macro form is used.
To use the function form, \#undef the macro name to uncover the
function.

\cstartmacro{void}{XSyncIntToValue}
\cargdecl{XSyncValue}{*pv},
\cargdecl{int}{i}
\cendmacrodecl

Converts \cargname{i} to an \ctypename{XSyncValue} and stores it in
\cargname{*pv}.  Performs sign extension (\cargname{*pv} will have the
same sign as \cargname{i}.)
\cendmacrodescription

\cstartmacro{void}{XSyncIntsToValue}
\cargdecl{XSyncValue}{*pv},
\cargdecl{unsigned int}{low},
\cargdecl{int}{high}
\cendmacrodecl

Stores \cargname{low} in the low 32 bits of \cargname{*pv} and 
\cargname{high} in the high 32 bits of \cargname{*pv}.
\cendmacrodescription

\cstartmacro{Bool}{XSyncValueGreaterThan}
\cargdecl{XSyncValue}{a},
\cargdecl{XSyncValue}{b}
\cendmacrodecl

Returns \cconst{True} if \cargname{a} is greater than \cargname{b},
else returns \cconst{False}.
\cendmacrodescription

\cstartmacro{Bool}{XSyncValueLessThan}
\cargdecl{XSyncValue}{a},
\cargdecl{XSyncValue}{b}
\cendmacrodecl

Returns \cconst{True} if \cargname{a} is less than \cargname{b},
else returns \cconst{False}.
\cendmacrodescription

\cstartmacro{Bool}{XSyncValueGreaterOrEqual}
\cargdecl{XSyncValue}{a},
\cargdecl{XSyncValue}{b}
\cendmacrodecl

Returns \cconst{True} if  \cargname{a} is greater than or equal to \cargname{b},
else returns \cconst{False}.
\cendmacrodescription

\cstartmacro{Bool}{XSyncValueLessOrEqual}
\cargdecl{XSyncValue}{a},
\cargdecl{XSyncValue}{b}
\cendmacrodecl

Returns \cconst{True} if  \cargname{a} is less than or equal to \cargname{b},
\cendmacrodescription

\cstartmacro{Bool}{XSyncValueEqual}
\cargdecl{XSyncValue}{a},
\cargdecl{XSyncValue}{b}
\cendmacrodecl

Returns \cconst{True} if  \cargname{a} is equal to \cargname{b},
else returns \cconst{False}.
\cendmacrodescription

\cstartmacro{Bool}{XSyncValueIsNegative}
\cargdecl{XSyncValue}{v}
\cendmacrodecl

Returns \cconst{True} if \cargname{v} is negative, else returns
\cconst{False}.
\cendmacrodescription

\cstartmacro{Bool}{XSyncValueIsZero}
\cargdecl{XSyncValue}{v}
\cendmacrodecl

Returns \cconst{True} if  \cargname{v} is zero,
else returns \cconst{False}.
\cendmacrodescription

\cstartmacro{Bool}{XSyncValueIsPositive}
\cargdecl{XSyncValue}{v}
\cendmacrodecl

Returns \cconst{True} if \cargname{v} is positive, else returns
\cconst{False}.
\cendmacrodescription

\cstartmacro{unsigned int}{XSyncValueLow32}
\cargdecl{XSyncValue}{v}
\cendmacrodecl

Returns the low 32 bits of \cargname{v}.
\cendmacrodescription

\cstartmacro{int}{XSyncValueHigh32}
\cargdecl{XSyncValue}{v}
\cendmacrodecl

Returns the high 32 bits of \cargname{v}.
\cendmacrodescription

\cstartmacro{void}{XSyncValueAdd}
\cargdecl{XSyncValue *}{presult},
\cargdecl{XSyncValue}{a},
\cargdecl{XSyncValue}{b},
\cargdecl{Bool *}{poverflow}
\cendmacrodecl

Adds \cargname{a} to \cargname{b} and stores the result in \cargname{*presult}.
If the result could not fit in 64 bits, \cargname{*poverflow} is set to
\cconst{True} else it is set to \cconst{False}.
\cendmacrodescription

\cstartmacro{void}{XSyncValueSubtract}
\cargdecl{XSyncValue *}{presult},
\cargdecl{XSyncValue}{a},
\cargdecl{XSyncValue}{b},
\cargdecl{Bool *}{poverflow}
\cendmacrodecl

Subtracts \cargname{b} from \cargname{a} and stores the result in
\cargname{*presult}.
If the result could not fit in 64 bits, \cargname{overflow} is set to
\cconst{True} else it is set to \cconst{False}.
\cendmacrodescription

\cstartmacro{void}{XSyncMaxValue}
\cargdecl{XSyncValue *}{pv}
\cendmacrodecl

Sets \cargname{*pv} to the maximum value expressible in 64 bits.
\cendmacrodescription

\cstartmacro{void}{XSyncMinValue}
\cargdecl{XSyncValue *}{pv}
\cendmacrodecl

Sets \cargname{*pv} to the minimum value expressible in 64 bits.
\cendmacrodescription

\subsection*{Events}

Let \cargname{event\_base} be the value \cargname{event\_base\_return}
as defined in the function \cfunctionname{XSyncQueryExtension}.

An \ctypename{XSyncCounterNotifyEvent}'s type field has the value
\cargname{event\_base} + \cconst{XSyncCounterNotify}.  The fields of
this structure are:

\begin{tabular}{lll}
\ctypename{int} & type;	& /* event base + \cconst{XSyncCounterNotify} */ \\
\ctypename{unsigned long} & serial; & /* number of last request processed by server */ \\
\ctypename{Bool} & send\_event;& /* true if this came from a SendEvent request */ \\
\ctypename{Display *} & display; & /* Display the event was read from */\\
\ctypename{XSyncCounter} & counter;	& /* counter involved in await */\\
\ctypename{XSyncValue} & wait\_value; & /* value being waited for */\\
\ctypename{XSyncValue} & counter\_value; & /* counter value when this event was sent */\\
\ctypename{Time} & time; & /* milliseconds */\\
\ctypename{int} & count; & /* how many more events to come */\\
\ctypename{Bool} & destroyed; & /* True if counter was destroyed */\\
\end{tabular}

An \ctypename{XSyncAlarmNotifyEvent}'s type field has the value
\cargname{event\_base} + \cconst{XSyncAlarmNotify}.  The fields of this
structure are:

\begin{tabular}{lll}
\ctypename{int} & type;&	/* event base + \cconst{XSyncAlarmNotify} */\\
\ctypename{unsigned long} & serial;&/* number of last request processed by server */\\
\ctypename{Bool} & send\_event;& /* true if this came from a SendEvent request */\\
\ctypename{Display *} & display;&	/* Display the event was read from */\\
\ctypename{XSyncAlarm} & alarm;&	/* alarm that triggered */\\
\ctypename{XSyncValue} & counter\_value;&/* value that triggered the alarm */\\
\ctypename{XSyncValue} & alarm\_value;&	/* test  value of trigger in alarm */\\
\ctypename{Time} & time;&	/* milliseconds */\\
\ctypename{XSyncAlarmState} & state;&	/* new state of alarm */\\
\end{tabular}

\subsection*{Errors}

Let \cargname{error\_base} be the value \cargname{error\_base\_return}
as defined in the function \cfunctionname{XSyncQueryExtension}.

An \ctypename{XSyncAlarmError}'s error\_code field has the value
\cargname{error\_base} + \cconst{XSyncBadAlarm}.  The fields of
this structure are:

\begin{tabular}{lll}
\ctypename{int} & type;	\\
\ctypename{Display *} & display;& /* Display the event was read from */\\
\ctypename{XSyncAlarm} &  alarm;& /* resource id */\\
\ctypename{unsigned long} & serial;& /* serial number of failed request */\\
\ctypename{unsigned char} & error\_code;&/* error base + XSyncBadAlarm */\\
\ctypename{unsigned char} & request\_code;&/* Major op-code of failed request */\\
\ctypename{unsigned char} & minor\_code;&/* Minor op-code of failed request */\\
\end{tabular}

An \ctypename{XSyncCounterError}'s error\_code field has the value
\cargname{error\_base} + \cconst{XSyncBadCounter}.  The fields of
this structure are:

\begin{tabular}{lll}
\ctypename{int} &type;\\
\ctypename{Display *} & display;&	/* Display the event was read from */\\
\ctypename{XSyncCounter} & counter;&	/* resource id */\\
\ctypename{unsigned long} & serial;&	/* serial number of failed request */\\
\ctypename{unsigned char} & error\_code;&/* error base + XSyncBadCounter */\\
\ctypename{unsigned char} & request\_code;&/* Major op-code of failed request */\\
\ctypename{unsigned char} & minor\_code;& /* Minor op-code of failed request */\\
\end{tabular}

\end{document}
