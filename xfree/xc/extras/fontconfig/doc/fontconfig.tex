\documentclass[10pt]{article}
\usepackage{latexsym}
\usepackage{epsfig}
\usepackage{times}

\begin{document}
\date{}
\title{The Fontconfig Library:\\
Architecture and Users Guide}
\author{Keith Packard\\
{\em XFree86 Core Team}\\
keithp@keithp.com}
\maketitle
\thispagestyle{empty}

\abstract

The Fontconfig library provides for central administration and configuration
of fonts in a POSIX system.  All font consumers can share a common database
of fonts and use common matching rules for font names.  The set of available
fonts can be configured for each user and a set of configurable matching
rules allow for customizing the selection of fonts and configuring various
parameters related to rasterizing of those fonts for display in a variety of
media.  The Fontconfig library is designed to co-exist peacefully with
existing font configuration and rasterization mechanisms; while it uses the
FreeType library to discover characteristics of available fonts, there
is no requirement to use FreeType for rasterization.

\section				{Introduction}

\section				{Configuration Files}

\section				{Application Interface}

\subsection				{Datatypes}

\subsection				{Font Set Interface}

\subsection				{Font Patterns}

\subsection				{Listing Available Fonts}

\subsection				{Using Font Names}

\subsection				{Manipulating Matrices}

\subsection				{UTF-8 Helper Functions}

\section				{Font Sub-System Interface}

\subsection				{Extending Font Names}

\subsection				{Executing Configuration Rules}

\end{document}
