% $Xorg: synclib.tex,v 1.3 2000/08/17 19:42:37 cpqbld Exp $
%
% Copyright 1991 by Olivetti Research Limited, Cambridge, England and
% Digital Equipment Corporation, Maynard, Massachusetts.
%
%                        All Rights Reserved
%
% Permission to use, copy, modify, and distribute this software and its 
% documentation for any purpose and without fee is hereby granted, 
% provided that the above copyright notice appear in all copies and that
% both that copyright notice and this permission notice appear in 
% supporting documentation, and that the names of Digital or Olivetti
% not be used in advertising or publicity pertaining to distribution of the
% software without specific, written prior permission.  
%
% DIGITAL AND OLIVETTI DISCLAIM ALL WARRANTIES WITH REGARD TO THIS SOFTWARE,
% INCLUDING ALL IMPLIED WARRANTIES OF MERCHANTABILITY AND FITNESS, IN NO EVENT
% SHALL THEY BE LIABLE FOR ANY SPECIAL, INDIRECT OR CONSEQUENTIAL DAMAGES OR
% ANY DAMAGES WHATSOEVER RESULTING FROM LOSS OF USE, DATA OR PROFITS, WHETHER
% IN AN ACTION OF CONTRACT, NEGLIGENCE OR OTHER TORTIOUS ACTION, ARISING OUT
% OF OR IN CONNECTION WITH THE USE OR PERFORMANCE OF THIS SOFTWARE.

%\documentstyle[a4]{article}
\documentstyle{article}

\setlength{\parindent}{0 pt}
\setlength{\parskip}{6pt}

% Protocol Section
% For the DP book, these four should be assigned the font for global symbols.

\newcommand{\request}[1]{{\bf #1}}
\newcommand{\event}[1]{{\bf #1}}
\newcommand{\error}[1]{{\bf #1}}
\newcommand{\enum}[1]{{\bf #1}}

% The following fonts are not reassigned for the DP book.

\newcommand{\system}[1]{{\sc #1}}
\newcommand{\param}[1]{{\it #1}}

\newcommand{\eventdef}[1]{\item {\bf#1}}
\newcommand{\requestdef}[1]{\item {\bf#1}}
\newcommand{\errordef}[1]{\item {\bf#1}}

\newcommand{\defn}[1]{{\bf #1}}

\newcommand{\tabstopsA}{\hspace*{4cm}\=\hspace*{1cm}\=\hspace*{7cm}\=\kill}
\newcommand{\tabstopsB}{\hspace*{1cm}\=\hspace*{1cm}\=\hspace*{3cm}\=\kill}
\newcommand{\tabstopsC}{\hspace*{1cm}\=\hspace*{1cm}\=\hspace*{5cm}\=\kill}

% commands for formatting the API
% For the DP book, these three should be assigned the font for global symbols.

\newcommand{\cfunctionname}[1]{\mbox{\tt#1}}
\newcommand{\ctypename}[1]{\mbox{\tt#1}}
\newcommand{\cconst}[1]{\mbox{\tt#1}}

% For the DP book, within function definitions, the type and name are in
% the ordinary font; therefore, ctypenamedef and cfunctionnamedef are used
% and defined below.
\newcommand{\ctypenamedef}[1]{\mbox{#1}}
\newcommand{\cfunctionnamedef}[1]{\mbox{#1}}
\newcommand{\cargname}[1]{\mbox{\it#1}}
\newcommand{\cstartfunction}[2]{\begin{sloppypar}\begin{samepage}\ctypenamedef{#1}\\ \cfunctionnamedef{#2}\ (}
\newcommand{\cargdecl}[2]{\penalty -1\ctypenamedef{#1} \cargname{#2}}
\newcommand{\cendfunctiondecl}{){\hangafter=2 \hangindent=20pt \raggedright\par}}
\newcommand{\cendfuncdescription}{\end{samepage}\end{sloppypar}}

\newcommand{\cstartmacro}[2]{\begin{sloppypar}\begin{samepage}\ctypenamedef{#1}\\ \cfunctionnamedef{#2}\ (}
\newcommand{\cendmacrodecl}{)\par}
\newcommand{\cendmacrodescription}{\end{samepage}\end{sloppypar}}

% make things easier with all the long names
\spaceskip .3333em plus 5em
\tolerance=2000

\begin{document}

\begin{center}

{\large X Synchronization Extension Library}\\[10pt]
{\large Version 3.0}\\[15pt]
{\large X Consortium Standard}\\[15pt]
{\large X Version 11, Release 6.4}\\[15pt]
{\it Tim Glauert}\\[0pt]
{\tt thg@cam-orl.co.uk}\\[0pt]
{\bf Olivetti Research / MultiWorks}\\[5pt]
{\it Dave Carver}\\[0pt]
{\tt dcc@athena.mit.edu}\\[0pt]
{\bf Digital Equipment Corporation,}\\[0pt]
{\bf MIT / Project Athena}\\[5pt]
{\it Jim Gettys}\\[0pt]
{\tt jg@crl.dec.com}\\[0pt]
{\bf Digital Equipment Corporation,}\\[0pt]
{\bf Cambridge Research Laboratory}\\[5pt]
{\it David P. Wiggins}\\[0pt]
{\tt dpw@x.org}\\[0pt]
{\bf X Consortium, Inc.}\\[0pt]

\end {center}

Copyright 1991 by Olivetti Research Limited, Cambridge, England and
Digital Equipment Corporation, Maynard, Massachusetts.

{\small Permission to use, copy, modify, and distribute this documentation
for any purpose and without fee is hereby granted, provided that the above
copyright notice appear in all copies. Olivetti, Digital, MIT, and the
X Consortium
make no representations about the suitability for any purpose of the
information in this document. This documentation is provided as is without
express or implied warranty.}

Copyright (c) 1991 X Consortium, Inc.

{\small Permission is hereby granted, free of charge, to any person obtaining a copy
of this software and associated documentation files (the ``Software''), to deal
in the Software without restriction, including without limitation the rights
to use, copy, modify, merge, publish, distribute, sublicense, and/or sell
copies of the Software, and to permit persons to whom the Software is
furnished to do so, subject to the following conditions:

The above copyright notice and this permission notice shall be included in
all copies or substantial portions of the Software.

THE SOFTWARE IS PROVIDED ``AS IS'', WITHOUT WARRANTY OF ANY KIND, EXPRESS OR
IMPLIED, INCLUDING BUT NOT LIMITED TO THE WARRANTIES OF MERCHANTABILITY,
FITNESS FOR A PARTICULAR PURPOSE AND NONINFRINGEMENT.  IN NO EVENT SHALL THE
X CONSORTIUM BE LIABLE FOR ANY CLAIM, DAMAGES OR OTHER LIABILITY, WHETHER IN
AN ACTION OF CONTRACT, TORT OR OTHERWISE, ARISING FROM, OUT OF OR IN
CONNECTION WITH THE SOFTWARE OR THE USE OR OTHER DEALINGS IN THE SOFTWARE.

Except as contained in this notice, the name of the X Consortium shall not be
used in advertising or otherwise to promote the sale, use or other dealings
in this Software without prior written authorization from the X Consortium.}
\eject

\section{Synchronization Protocol}

The core X protocol makes no guarantees about the relative order of execution
of requests for different clients. This means that any synchronization between
clients must be done at the client level in an operating system-dependent and
network-dependent manner. Even if there was an accepted standard for such
synchronization, the use of a network introduces unpredictable delays between
the synchronization of the clients and the delivery of the resulting requests
to the X server.

The core X protocol also makes no guarantees about the time at which requests
are executed, which means that all clients with real-time constraints must
implement their timing on the host computer. Any such timings are subject to
error introduced by delays within the operating system and network and are
inefficient because of the need for round-trip requests that keep the client and
server synchronized.

The synchronization extension provides primitives that allow synchronization
between clients to take place entirely within the X server. This removes any
error introduced by the network and makes it possible to synchronize clients
on different hosts running different operating systems. This is important for
multimedia applications, where audio, video, and graphics data streams are
being synchronized. The extension also provides internal timers within the X
server to which client requests can be synchronized. This allows simple
animation applications to be implemented without any round-trip requests and
makes best use of buffering within the client, network, and server.

\subsection{Description}

The mechanism used by this extension for synchronization within the X server
is to block the processing of requests from a client until a specific
synchronization condition occurs. When the condition occurs, the client is
released and processing of requests continues. Multiple clients may block on
the same condition to give inter-client synchronization.  Alternatively, a
single client may block on a condition such as an animation frame marker.

The extension adds \defn{Counter} and \defn{Alarm} to the set of resources
managed by the server. A counter has a 64-bit integer value that may be
increased or decreased by client requests or by the server internally. A
client can block by sending an \request{Await} request that waits until
one of a set of synchronization conditions, called TRIGGERs, becomes TRUE.

The \request{CreateCounter} request allows a client to create a
\defn{Counter} that can be changed by explicit \request{SetCounter} and
\request{ChangeCounter} requests. These can be used to implement
synchronization between different clients.

There are some counters, called \defn{System Counters}, that are changed by
the server internally rather than by client requests. The effect of any change
to a system counter is not visible until the server has finished processing the
current request. In other words, system counters are apparently updated in the
gaps between the execution of requests rather than during the actual execution
of a request. The extension provides a system counter that advances with the
server time as defined by the core protocol, and it may also provide counters
that advance with the real-world time or that change each time the CRT
screen is refreshed.  Other extensions may provide their own
extension-specific system counters.

The extension provides an \defn{Alarm} mechanism that allows clients to
receive an event on a regular basis when a particular counter is changed.

\section{C Language Binding}

The C routines provide direct access to	the protocol and add
no additional semantics.

The include file for this extension is \verb|<X11/extensions/sync.h>|.

Most of the names in the language binding are derived from the
protocol names by prepending \cfunctionname{XSync} to the protocol name and changing
the capitalization.

\subsection{C Functions}

Most of the following functions generate SYNC protocol requests.

% start marker
\cstartfunction{Status}{XSyncQueryExtension}
\cargdecl{Display *}{dpy},
\cargdecl{int *}{event\_base\_return},
\cargdecl{int *}{error\_base\_return}
\cendfunctiondecl
% end marker

If dpy supports the SYNC extension, \cfunctionname{XSyncQueryExtension}
returns \cconst{True}, sets *event\_base\_return to the event number for the
first SYNC event, and sets
*error\_base\_return to the error number for the first SYNC
error.  If dpy does not support the SYNC extension, it
returns \cconst{False}.
\cendfuncdescription


% start marker
\cstartfunction{Status}{XSyncInitialize}
\cargdecl{Display *}{dpy},
\cargdecl{int *}{major\_version\_return},
\cargdecl{int *}{minor\_version\_return}
\cendfunctiondecl
% end marker

\cfunctionname{XSyncInitialize} sets *major\_version\_return and
*minor\_version\_return to the major/minor SYNC protocol
version supported by the server.  If the XSync library is compatible
with the version returned by the server, this function returns \cconst{True}.
If dpy does not support the SYNC extension, or if there was an error during
communication with the server, or if the server and library protocol
versions are incompatible, this function returns \cconst{False}.  The only
XSync function that may be called before this function is
\cfunctionname{XSyncQueryExtension}.  If a client violates this rule,
the effects of all XSync calls that it makes are undefined.
\cendfuncdescription


% start marker
\cstartfunction{XSyncSystemCounter *}{XSyncListSystemCounters}
\cargdecl{Display *}{dpy},
\cargdecl{int *}{n\_counters\_return}
\cendfunctiondecl
% end marker

\cfunctionname{XSyncListSystemCounters} returns a pointer to an
array of system counters supported by the display and sets 
*n\_counters\_return to the number of
counters in the array.  The array should be freed with
\cfunctionname{XSyncFreeSystemCounterList}.  If dpy does not
support the SYNC extension,
or if there was an error during communication with the
server, or if the server does not support any system counters, this
function returns NULL.

\ctypename{XSyncSystemCounter} has the following fields:

\begin{tabular}{lll}
char * & name; & /* null-terminated name of system counter */\\
XSyncCounter & counter; & /* counter id of this system counter */\\
XSyncValue & resolution; & /* resolution of this system counter */\\
\end{tabular}
\cendfuncdescription


% start marker
\cstartfunction{void}{XSyncFreeSystemCounterList}
\cargdecl{XSyncSystemCounter *}{list}
\cendfunctiondecl
% end marker

\cfunctionname{XSyncFreeSystemCounterList} frees the memory associated 
with the system counter list returned by \cfunctionname{XSyncListSystemCounters}.
\cendfuncdescription


% start marker
\cstartfunction{XSyncCounter}{XSyncCreateCounter}
\cargdecl{Display *}{dpy}, 
\cargdecl{XSyncValue}{initial\_value}
\cendfunctiondecl
% end marker

\cfunctionname{XSyncCreateCounter} creates a counter on the dpy
with the given initial\_value and returns the counter ID. 
It returns \cconst{None} if dpy does not
support the SYNC extension.
\cendfuncdescription


% start marker
\cstartfunction{Status}{XSyncSetCounter}
\cargdecl{Display *}{dpy},
\cargdecl{XSyncCounter}{counter},
\cargdecl{XSyncValue}{value}
\cendfunctiondecl
% end marker

\cfunctionname{XSyncSetCounter} sets counter to value.  
It returns \cconst{False} if dpy does not support the SYNC extension;
otherwise, it returns \cconst{True}.
\cendfuncdescription


% start marker
\cstartfunction{Status}{XSyncChangeCounter}
\cargdecl{Display *}{dpy},
\cargdecl{XSyncCounter}{counter},
\cargdecl{XSyncValue}{value}
\cendfunctiondecl
% end marker

\cfunctionname{XSyncChangeCounter} adds value to counter.
It returns \cconst{False} if dpy does not support the SYNC extension;
otherwise, it returns \cconst{True}.
\cendfuncdescription


% start marker
\cstartfunction{Status}{XSyncDestroyCounter}
\cargdecl{Display *}{dpy}, 
\cargdecl{XSyncCounter}{counter}
\cendfunctiondecl
% end marker

\cfunctionname{XSyncDestroyCounter} destroys counter. 
It returns \cconst{False} if dpy does not
support the SYNC extension; otherwise, it returns \cconst{True}.
\cendfuncdescription


% start marker
\cstartfunction{Status}{XSyncQueryCounter}
\cargdecl{Display *}{dpy},
\cargdecl{XSyncCounter}{counter},
\cargdecl{XSyncValue *}{value\_return}
\cendfunctiondecl
% end marker

\cfunctionname{XSyncQueryCounter} sets *value\_return to the current value of
counter.  It returns \cconst{False} if there was an error during
communication with the server or if dpy does not support
the SYNC extension; otherwise, it returns \cconst{True}.
\cendfuncdescription


% start marker
\cstartfunction{Status}{XSyncAwait}
\cargdecl{Display *}{dpy},
\cargdecl{XSyncWaitCondition *}{wait\_list},
\cargdecl{int}{n\_conditions}
\cendfunctiondecl
% end marker

\cfunctionname{XSyncAwait} awaits on the conditions in wait\_list.
The n\_conditions is the number of wait conditions in
wait\_list.  It returns \cconst{False} if dpy does not
support the SYNC extension; otherwise, it returns \cconst{True}.
The await is processed asynchronously by the server;
this function always returns immediately
after issuing the request.

\ctypename{XSyncWaitCondition} has the following fields:

\begin{tabular}{lll}
XSyncCounter & trigger.counter; & /* counter to trigger on */ \\
XSyncValueType & trigger.value\_type; & /* absolute/relative */ \\
XSyncValue & trigger.wait\_value; & /* value to compare counter to */ \\
XSyncTestType & trigger.test\_type;	& /* pos/neg comparison/transtion */ \\
XSyncValue & event\_threshold; & /* send event if past threshold */ \\
\end{tabular}

\ctypename{XSyncValueType} can be either \cconst{XSyncAbsolute} or \cconst{XSyncRelative}.

\ctypename{XSyncTestType} can be one of \cconst{XSyncPositiveTransition},
\cconst{XSyncNegativeTransition}, \cconst{XSyncPositiveComparison}, or
\cconst{XSyncNegativeComparison}.
\cendfuncdescription


% start marker
\cstartfunction{XSyncAlarm}{XSyncCreateAlarm}
\cargdecl{Display *}{dpy},
\cargdecl{unsigned long}{values\_mask},
\cargdecl{XSyncAlarmAttributes *}{values}
\cendfunctiondecl
% end marker

\cfunctionname{XSyncCreateAlarm} creates an alarm and returns the alarm ID.
It returns \cconst{None} if the display does not support the SYNC extension. 
The values\_mask and values specify the alarm attributes.

\ctypename{XSyncAlarmAttributes} has the following fields.  The attribute\_mask
column specifies the symbol that the caller should OR into
values\_mask to indicate that the value for the corresponding
attribute was actually supplied.  Default values are used for all
attributes that do not have their attribute\_mask OR'ed into 
values\_mask.
See the protocol
description for \request{CreateAlarm} for the defaults.

\begin{tabular}{lll}
type & field name & attribute\_mask \\
XSyncCounter & trigger.counter; & \cconst{XSyncCACounter} \\
XSyncValueType & trigger.value\_type; & \cconst{XSyncCAValueType} \\
XSyncValue & trigger.wait\_value; & \cconst{XSyncCAValue} \\
XSyncTestType & trigger.test\_type; & \cconst{XSyncCATestType} \\
XSyncValue & delta;	& \cconst{XSyncCADelta} \\
Bool & events; & \cconst{XSyncCAEvents} \\
XSyncAlarmState & state; & client cannot set this \\
\end{tabular}
\cendfuncdescription


% start marker
\cstartfunction{Status}{XSyncDestroyAlarm}
\cargdecl{Display *}{dpy},
\cargdecl{XSyncAlarm}{alarm}
\cendfunctiondecl
% end marker

\cfunctionname{XSyncDestroyAlarm} destroys alarm. 
It returns \cconst{False} if dpy does not
support the SYNC extension; otherwise, it returns \cconst{True}.
\cendfuncdescription


% start marker
\cstartfunction{Status}{XSyncQueryAlarm}
\cargdecl{Display *}{dpy},
\cargdecl{XSyncAlarm}{alarm},
\cargdecl{XSyncAlarmAttributes *}{values\_return}
\cendfunctiondecl
% end marker

\cfunctionname{XSyncQueryAlarm} sets *values\_return to the alarm's
attributes.  It returns \cconst{False} if there was an error 
during communication with the server or if dpy does not support 
the SYNC extension; otherwise, it returns \cconst{True}.
\cendfuncdescription


% start marker
\cstartfunction{Status}{XSyncChangeAlarm}
\cargdecl{Display *}{dpy},
\cargdecl{XSyncAlarm}{alarm},
\cargdecl{unsigned long}{values\_mask},
\cargdecl{XSyncAlarmAttributes *}{values}
\cendfunctiondecl
% end marker

\cfunctionname{XSyncChangeAlarm} changes alarm's attributes.
The attributes to change are specified as in \cfunctionname{XSyncCreateAlarm}.
It returns \cconst{False} if dpy does not support the SYNC extension; 
otherwise, it returns \cconst{True}.
\cendfuncdescription


% start marker
\cstartfunction{Status}{XSyncSetPriority}
\cargdecl{Display *}{dpy},
\cargdecl{XID}{client\_resource\_id},
\cargdecl{int}{priority}
\cendfunctiondecl
% end marker

\cfunctionname{XSyncSetPriority} sets the priority of the client owning 
client\_resource\_id to priority.  
If client\_resource\_id is \cconst{None},
it sets the caller's priority.  It returns \cconst{False} if dpy
does not support the SYNC extension; otherwise, it returns \cconst{True}.
\cendfuncdescription


% start marker
\cstartfunction{Status}{XSyncGetPriority}
\cargdecl{Display *}{dpy},
\cargdecl{XID}{client\_resource\_id},
\cargdecl{int *}{return\_priority}
\cendfunctiondecl
% end marker

\cfunctionname{XSyncGetPriority} sets *return\_priority to the priority
of the client owning client\_resource\_id. 
If client\_resource\_id is
\cconst{None}, it sets *return\_priority to the caller's priority.
It returns \cconst{False} if there was an error
during communication with the server or if dpy does not
support the SYNC extension; otherwise, it returns \cconst{True}.
\cendfuncdescription

\subsection{C Macros/Functions}

The following procedures manipulate 64-bit values.  They are defined
both as macros and as functions.  By default, the macro form is used.
To use the function form, \#undef the macro name to uncover the
function.


% start marker
\cstartmacro{void}{XSyncIntToValue}
\cargdecl{XSyncValue}{*pv},
\cargdecl{int}{i}
\cendmacrodecl
% end marker

Converts i to an \ctypename{XSyncValue} and stores it in
*pv.  Performs sign extension (*pv will have the
same sign as i.)
\cendmacrodescription


% start marker
\cstartmacro{void}{XSyncIntsToValue}
\cargdecl{XSyncValue}{*pv},
\cargdecl{unsigned int}{low},
\cargdecl{int}{high}
\cendmacrodecl
% end marker

Stores low in the low 32 bits of *pv and 
high in the high 32 bits of *pv.
\cendmacrodescription


% start marker
\cstartmacro{Bool}{XSyncValueGreaterThan}
\cargdecl{XSyncValue}{a},
\cargdecl{XSyncValue}{b}
\cendmacrodecl
% end marker

Returns \cconst{True} if a is greater than b,
else returns \cconst{False}.
\cendmacrodescription


% start marker
\cstartmacro{Bool}{XSyncValueLessThan}
\cargdecl{XSyncValue}{a},
\cargdecl{XSyncValue}{b}
\cendmacrodecl
% end marker

Returns \cconst{True} if a is less than b,
else returns \cconst{False}.
\cendmacrodescription


% start marker
\cstartmacro{Bool}{XSyncValueGreaterOrEqual}
\cargdecl{XSyncValue}{a},
\cargdecl{XSyncValue}{b}
\cendmacrodecl
% end marker

Returns \cconst{True} if a is greater than or equal to b,
else returns \cconst{False}.
\cendmacrodescription


% start marker
\cstartmacro{Bool}{XSyncValueLessOrEqual}
\cargdecl{XSyncValue}{a},
\cargdecl{XSyncValue}{b}
\cendmacrodecl
% end marker

Returns \cconst{True} if a is less than or equal to b.
\cendmacrodescription


% start marker
\cstartmacro{Bool}{XSyncValueEqual}
\cargdecl{XSyncValue}{a},
\cargdecl{XSyncValue}{b}
\cendmacrodecl
% end marker

Returns \cconst{True} if a is equal to b,
else returns \cconst{False}.
\cendmacrodescription


% start marker
\cstartmacro{Bool}{XSyncValueIsNegative}
\cargdecl{XSyncValue}{v}
\cendmacrodecl
% end marker

Returns \cconst{True} if v is negative, else returns
\cconst{False}.
\cendmacrodescription


% start marker
\cstartmacro{Bool}{XSyncValueIsZero}
\cargdecl{XSyncValue}{v}
\cendmacrodecl
% end marker

Returns \cconst{True} if v is zero,
else returns \cconst{False}.
\cendmacrodescription


% start marker
\cstartmacro{Bool}{XSyncValueIsPositive}
\cargdecl{XSyncValue}{v}
\cendmacrodecl
% end marker

Returns \cconst{True} if v is positive, else returns
\cconst{False}.
\cendmacrodescription


% start marker
\cstartmacro{unsigned int}{XSyncValueLow32}
\cargdecl{XSyncValue}{v}
\cendmacrodecl
% end marker

Returns the low 32 bits of v.
\cendmacrodescription


% start marker
\cstartmacro{int}{XSyncValueHigh32}
\cargdecl{XSyncValue}{v}
\cendmacrodecl
% end marker

Returns the high 32 bits of v.
\cendmacrodescription


% start marker
\cstartmacro{void}{XSyncValueAdd}
\cargdecl{XSyncValue *}{presult},
\cargdecl{XSyncValue}{a},
\cargdecl{XSyncValue}{b},
\cargdecl{Bool *}{poverflow}
\cendmacrodecl
% end marker

Adds a to b and stores the result in *presult.
If the result could not fit in 64 bits, *poverflow is set to
\cconst{True}, else it is set to \cconst{False}.
\cendmacrodescription


% start marker
\cstartmacro{void}{XSyncValueSubtract}
\cargdecl{XSyncValue *}{presult},
\cargdecl{XSyncValue}{a},
\cargdecl{XSyncValue}{b},
\cargdecl{Bool *}{poverflow}
\cendmacrodecl
% end marker

Subtracts b from a and stores the result in
*presult.
If the result could not fit in 64 bits, overflow is set to
\cconst{True}, else it is set to \cconst{False}.
\cendmacrodescription


% start marker
\cstartmacro{void}{XSyncMaxValue}
\cargdecl{XSyncValue *}{pv}
\cendmacrodecl
% end marker

Sets *pv to the maximum value expressible in 64 bits.
\cendmacrodescription


% start marker
\cstartmacro{void}{XSyncMinValue}
\cargdecl{XSyncValue *}{pv}
\cendmacrodecl
% end marker

Sets *pv to the minimum value expressible in 64 bits.
\cendmacrodescription

\subsection{Events}

Let \cargname{event\_base} be the value \cargname{event\_base\_return}
as defined in the function \cfunctionname{XSyncQueryExtension}.

An \ctypename{XSyncCounterNotifyEvent}'s type field has the value
\cargname{event\_base} + \cconst{XSyncCounterNotify}.  The fields of
this structure are:

\begin{tabular}{lll}
int & type;	& /* event base + \cconst{XSyncCounterNotify} */ \\
unsigned long & serial; & /* number of last request processed by server */ \\
Bool & send\_event;& /* true if this came from a SendEvent request */ \\
Display * & display; & /* Display the event was read from */\\
XSyncCounter & counter;	& /* counter involved in await */\\
XSyncValue & wait\_value; & /* value being waited for */\\
XSyncValue & counter\_value; & /* counter value when this event was sent */\\
Time & time; & /* milliseconds */\\
int & count; & /* how many more events to come */\\
Bool & destroyed; & /* True if counter was destroyed */\\
\end{tabular}

An \ctypename{XSyncAlarmNotifyEvent}'s type field has the value
\cargname{event\_base} + \cconst{XSyncAlarmNotify}.  The fields of this
structure are:

\begin{tabular}{lll}
int & type;&	/* event base + \cconst{XSyncAlarmNotify} */\\
unsigned long & serial;&/* number of last request processed by server */\\
Bool & send\_event;& /* true if this came from a SendEvent request */\\
Display * & display;&	/* Display the event was read from */\\
XSyncAlarm & alarm;&	/* alarm that triggered */\\
XSyncValue & counter\_value;&/* value that triggered the alarm */\\
XSyncValue & alarm\_value;&	/* test  value of trigger in alarm */\\
Time & time;&	/* milliseconds */\\
XSyncAlarmState & state;&	/* new state of alarm */\\
\end{tabular}

\subsection{Errors}

Let \cargname{error\_base} be the value \cargname{error\_base\_return}
as defined in the function \cfunctionname{XSyncQueryExtension}.

An \ctypename{XSyncAlarmError}'s error\_code field has the value
\cargname{error\_base} + \cconst{XSyncBadAlarm}.  The fields of
this structure are:

\begin{tabular}{lll}
int & type;	\\
Display * & display;& /* Display the event was read from */\\
XSyncAlarm &  alarm;& /* resource id */\\
unsigned long & serial;& /* serial number of failed request */\\
unsigned char & error\_code;&/* error base + XSyncBadAlarm */\\
unsigned char & request\_code;&/* Major op-code of failed request */\\
unsigned char & minor\_code;&/* Minor op-code of failed request */\\
\end{tabular}

An \ctypename{XSyncCounterError}'s error\_code field has the value
\cargname{error\_base} + \cconst{XSyncBadCounter}.  The fields of
this structure are:

\begin{tabular}{lll}
int &type;\\
Display * & display;&	/* Display the event was read from */\\
XSyncCounter & counter;&	/* resource id */\\
unsigned long & serial;&	/* serial number of failed request */\\
unsigned char & error\_code;&/* error base + XSyncBadCounter */\\
unsigned char & request\_code;&/* Major op-code of failed request */\\
unsigned char & minor\_code;& /* Minor op-code of failed request */\\
\end{tabular}

\end{document}
