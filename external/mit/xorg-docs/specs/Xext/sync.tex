% $Xorg: sync.tex,v 1.3 2000/08/17 19:42:37 cpqbld Exp $
% $XdotOrg: xc/doc/specs/Xext/sync.tex,v 1.2 2004/04/23 18:42:18 eich Exp $
%
% Copyright 1991 by Olivetti Research Limited, Cambridge, England and
% Digital Equipment Corporation, Maynard, Massachusetts.
%
%                        All Rights Reserved
%
% Permission to use, copy, modify, and distribute this software and its 
% documentation for any purpose and without fee is hereby granted, 
% provided that the above copyright notice appear in all copies and that
% both that copyright notice and this permission notice appear in 
% supporting documentation, and that the names of Digital or Olivetti
% not be used in advertising or publicity pertaining to distribution of the
% software without specific, written prior permission.  
%
% DIGITAL AND OLIVETTI DISCLAIM ALL WARRANTIES WITH REGARD TO THIS SOFTWARE,
% INCLUDING ALL IMPLIED WARRANTIES OF MERCHANTABILITY AND FITNESS, IN NO EVENT
% SHALL THEY BE LIABLE FOR ANY SPECIAL, INDIRECT OR CONSEQUENTIAL DAMAGES OR
% ANY DAMAGES WHATSOEVER RESULTING FROM LOSS OF USE, DATA OR PROFITS, WHETHER
% IN AN ACTION OF CONTRACT, NEGLIGENCE OR OTHER TORTIOUS ACTION, ARISING OUT
% OF OR IN CONNECTION WITH THE USE OR PERFORMANCE OF THIS SOFTWARE.
%
% $XFree86$

%\documentstyle[a4]{article}
\documentstyle{article}

\setlength{\parindent}{0 pt}
\setlength{\parskip}{6pt}

% Protocol Section
% For the DP book, these four should be assigned the font for global symbols.

\newcommand{\request}[1]{{\bf #1}}
\newcommand{\event}[1]{{\bf #1}}
\newcommand{\error}[1]{{\bf #1}}
\newcommand{\enum}[1]{{\bf #1}}

% The following fonts are not reassigned for the DP book.

\newcommand{\system}[1]{{\sc #1}}
\newcommand{\param}[1]{{\it #1}}

\newcommand{\eventdef}[1]{\item {\bf#1}}
\newcommand{\requestdef}[1]{\item {\bf#1}}
\newcommand{\errordef}[1]{\item {\bf#1}}

\newcommand{\defn}[1]{{\bf #1}}

\newcommand{\tabstopsA}{\hspace*{4cm}\=\hspace*{1cm}\=\hspace*{7cm}\=\kill}
\newcommand{\tabstopsB}{\hspace*{1cm}\=\hspace*{1cm}\=\hspace*{3cm}\=\kill}
\newcommand{\tabstopsC}{\hspace*{1cm}\=\hspace*{1cm}\=\hspace*{5cm}\=\kill}

% commands for formatting the API
% For the DP book, these three should be assigned the font for global symbols.

\newcommand{\cfunctionname}[1]{\mbox{\tt#1}}
\newcommand{\ctypename}[1]{\mbox{\tt#1}}
\newcommand{\cconst}[1]{\mbox{\tt#1}}

% For the DP book, within function definitions, the type and name are in
% the ordinary font; therefore, ctypenamedef and cfunctionnamedef are used
% and defined below.
\newcommand{\ctypenamedef}[1]{\mbox{#1}}
\newcommand{\cfunctionnamedef}[1]{\mbox{#1}}
\newcommand{\cargname}[1]{\mbox{\it#1}}
\newcommand{\cstartfunction}[2]{\begin{sloppypar}\begin{samepage}\ctypenamedef{#1}\\ \cfunctionnamedef{#2}\ (}
\newcommand{\cargdecl}[2]{\penalty -1\ctypenamedef{#1} \cargname{#2}}
\newcommand{\cendfunctiondecl}{){\hangafter=2 \hangindent=20pt \raggedright\par}}
\newcommand{\cendfuncdescription}{\end{samepage}\end{sloppypar}}

\newcommand{\cstartmacro}[2]{\begin{sloppypar}\begin{samepage}\ctypenamedef{#1}\\ \cfunctionnamedef{#2}\ (}
\newcommand{\cendmacrodecl}{)\par}
\newcommand{\cendmacrodescription}{\end{samepage}\end{sloppypar}}

% make things easier with all the long names
\spaceskip .3333em plus 5em
\tolerance=2000

\begin{document}

\begin{center}

{\large X Synchronization Extension Protocol}\\[10pt]
{\large Version 3.0}\\[15pt]
{\large X Consortium Standard}\\[15pt]
{\large X Version 11, Release 6.8}\\[15pt]
{\it Tim Glauert}\\[0pt]
{\tt thg@cam-orl.co.uk}\\[0pt]
{\bf Olivetti Research / MultiWorks}\\[5pt]
{\it Dave Carver}\\[0pt]
{\tt dcc@athena.mit.edu}\\[0pt]
{\bf Digital Equipment Corporation,}\\[0pt]
{\bf MIT / Project Athena}\\[5pt]
{\it Jim Gettys}\\[0pt]
{\tt jg@crl.dec.com}\\[0pt]
{\bf Digital Equipment Corporation,}\\[0pt]
{\bf Cambridge Research Laboratory}\\[5pt]
{\it David P. Wiggins}\\[0pt]
{\tt dpw@x.org}\\[0pt]
{\bf X Consortium, Inc.}\\[0pt]

\end {center}

Copyright 1991 by Olivetti Research Limited, Cambridge, England and
Digital Equipment Corporation, Maynard, Massachusetts.

{\small Permission to use, copy, modify, and distribute this documentation
for any purpose and without fee is hereby granted, provided that the above
copyright notice appear in all copies. Olivetti, Digital, MIT, and the
X Consortium
make no representations about the suitability for any purpose of the
information in this document. This documentation is provided as is without
express or implied warranty.}

Copyright (c) 1991 X Consortium, Inc.

{\small Permission is hereby granted, free of charge, to any person obtaining a copy
of this software and associated documentation files (the ``Software''), to deal
in the Software without restriction, including without limitation the rights
to use, copy, modify, merge, publish, distribute, sublicense, and/or sell
copies of the Software, and to permit persons to whom the Software is
furnished to do so, subject to the following conditions:

The above copyright notice and this permission notice shall be included in
all copies or substantial portions of the Software.

THE SOFTWARE IS PROVIDED ``AS IS'', WITHOUT WARRANTY OF ANY KIND, EXPRESS OR
IMPLIED, INCLUDING BUT NOT LIMITED TO THE WARRANTIES OF MERCHANTABILITY,
FITNESS FOR A PARTICULAR PURPOSE AND NONINFRINGEMENT.  IN NO EVENT SHALL THE
X CONSORTIUM BE LIABLE FOR ANY CLAIM, DAMAGES OR OTHER LIABILITY, WHETHER IN
AN ACTION OF CONTRACT, TORT OR OTHERWISE, ARISING FROM, OUT OF OR IN
CONNECTION WITH THE SOFTWARE OR THE USE OR OTHER DEALINGS IN THE SOFTWARE.

Except as contained in this notice, the name of the X Consortium shall not be
used in advertising or otherwise to promote the sale, use or other dealings
in this Software without prior written authorization from the X Consortium.}
\eject

\section{Synchronization Protocol}

The core X protocol makes no guarantees about the relative order of execution
of requests for different clients. This means that any synchronization between
clients must be done at the client level in an operating system-dependent and
network-dependent manner. Even if there was an accepted standard for such
synchronization, the use of a network introduces unpredictable delays between
the synchronization of the clients and the delivery of the resulting requests
to the X server.

The core X protocol also makes no guarantees about the time at which requests
are executed, which means that all clients with real-time constraints must
implement their timing on the host computer. Any such timings are subject to
error introduced by delays within the operating system and network and are
inefficient because of the need for round-trip requests that keep the client and
server synchronized.

The synchronization extension provides primitives that allow synchronization
between clients to take place entirely within the X server. This removes any
error introduced by the network and makes it possible to synchronize clients
on different hosts running different operating systems. This is important for
multimedia applications, where audio, video, and graphics data streams are
being synchronized. The extension also provides internal timers within the X
server to which client requests can be synchronized. This allows simple
animation applications to be implemented without any round-trip requests and
makes best use of buffering within the client, network, and server.

\subsection{Description}

The mechanism used by this extension for synchronization within the X server
is to block the processing of requests from a client until a specific
synchronization condition occurs. When the condition occurs, the client is
released and processing of requests continues. Multiple clients may block on
the same condition to give inter-client synchronization.  Alternatively, a
single client may block on a condition such as an animation frame marker.

The extension adds \defn{Counter} and \defn{Alarm} to the set of resources
managed by the server. A counter has a 64-bit integer value that may be
increased or decreased by client requests or by the server internally. A
client can block by sending an \request{Await} request that waits until
one of a set of synchronization conditions, called TRIGGERs, becomes TRUE.

The \request{CreateCounter} request allows a client to create a
\defn{Counter} that can be changed by explicit \request{SetCounter} and
\request{ChangeCounter} requests. These can be used to implement
synchronization between different clients.

There are some counters, called \defn{System Counters}, that are changed by
the server internally rather than by client requests. The effect of any change
to a system counter is not visible until the server has finished processing the
current request. In other words, system counters are apparently updated in the
gaps between the execution of requests rather than during the actual execution
of a request. The extension provides a system counter that advances with the
server time as defined by the core protocol, and it may also provide counters
that advance with the real-world time or that change each time the CRT
screen is refreshed.  Other extensions may provide their own
extension-specific system counters.

The extension provides an \defn{Alarm} mechanism that allows clients to
receive an event on a regular basis when a particular counter is changed.

\subsection{Types}

Please refer to the X11 Protocol specification as this document uses
syntactic conventions established there and references types defined there.

The following new types are used by the extension.

\begin{tabbing}{l}
SYSTEMCOUNTER: \=\kill
	INT64:	\>64-bit signed integer\\
	COUNTER:\>XID\\
  	VALUETYPE:\>  \{\enum{Absolute},\enum{Relative}\}\\
 	TESTTYPE:\> \{\enum{PositiveTransition},\enum{NegativeTransition},\\
		\>\enum{PositiveComparison},\enum{NegativeComparison}\}\\
	TRIGGER:\>[\\
 		\>counter:COUNTER,\\
		\>value-type:VALUETYPE,\\
		\>wait-value:INT64,\\
		\>test-type:TESTTYPE\\
		\>]\\
	WAITCONDITION:\>[\\
		\>trigger:TRIGGER,\\
		\>event-threshold:INT64\\
		\>]\\
	SYSTEMCOUNTER:\>[\\
		\>name:STRING8,\\
		\>counter:COUNTER,\\
		\>resolution:INT64\\
		\>]\\
	ALARM:	\>XID\\
	ALARMSTATE:\> \{\enum{Active},\enum{Inactive},\enum{Destroyed}\}\\
\end{tabbing}

The COUNTER type defines the client-side handle on a server \defn{Counter}.
The value of a counter is an INT64.

The TRIGGER type defines a test on a counter that is either TRUE or FALSE.
The value of the test is determined by the combination of a test value, the
value of the counter, and the specified test-type.

The test value for a trigger is calculated using the value-type and
wait-value fields when the trigger is initialized. 
If the value-type field is not one of the 
named VALUETYPE constants, the request that initialized the trigger
will return a \error{Value} error. If the
value-type field is \enum{Absolute}, the test value is given by the
wait-value field. If the value-type field is
\enum{Relative}, the test value is obtained by adding the
wait-value field to the value of the counter.  If the
resulting test value would lie outside the range for an INT64, the
request that initialized the trigger will return a
\error{Value} error. If counter is \enum{None} and the
value-type is \enum{Relative}, the request that initialized the 
trigger will return a \error{Match} error. 
If counter is not \enum{None} and does not name a valid
counter, a \error{Counter} error is generated.

If the test-type is \enum{PositiveTransition}, the trigger is
initialized to FALSE, and it will become TRUE when the counter changes
from a value less than the test value to a value greater than or equal to the
test value. If the test-type is \enum{NegativeTransition}, the
trigger is initialize to FALSE, and it will become TRUE when the
counter changes from a value greater than the test value to a value
less than or equal to the test value. If the test-type is
\enum{PositiveComparison}, the trigger is TRUE if the counter is greater than or
equal to the test value and FALSE otherwise.  If the test-type is
\enum{NegativeComparison}, the trigger is TRUE if the counter is less than or
equal to the test value and FALSE otherwise. If the test-type
is not one of the named TESTTYPE constants, the request that
initialized the trigger will return a \error{Value} error.  A trigger
with a counter value of \enum{None} and a valid test-type
is always TRUE.

The WAITCONDITION type is simply a trigger with an associated
event-threshold.  The event threshold is used by the \request{Await}
request to decide whether or not to generate an event to the client after the
trigger has become TRUE. By setting the event-threshold to an
appropriate value, it is possible to detect the situation where an
\request{Await} request was processed after the TRIGGER became TRUE,
which usually indicates that the server is not processing requests as fast as
the client expects.

The SYSTEMCOUNTER type provides the client with information about a
\defn{System Counter}. The name field is the textual name of the
counter that identifies the counter to the client. The counter field
is the client-side handle that should be used in requests that require a
counter. The resolution field gives the approximate step size of the
system counter. This is a hint to the client that the extension may not be
able to resolve two wait conditions with test values that differ by less than
this step size. A microsecond clock, for example, may advance in steps of 64
microseconds, so a counter based on this clock would have a resolution
of 64.

The only system counter that is guaranteed to be present is called
\system{SERVERTIME}, which counts milliseconds from some arbitrary starting
point. The least significant 32 bits of this counter track the value of Time
used by the server in Events and Requests. Other system counters may be
provided by different implementations of the extension. The X Consortium will
maintain a registry of system counter names to avoid collisions in the
name space.

An ALARM is the client-side handle on an \defn{Alarm} resource.

\subsection{Errors}

\begin{description}

\errordef{Counter}

This error is generated if the value for a COUNTER argument in a request does
not name a defined COUNTER.

\errordef{Alarm}

This error is generated if the value for an ALARM argument in a request does
not name a defined ALARM.

\end{description}

\subsection{Requests}

\begin{description}

% start marker
\requestdef{Initialize}

\begin{tabular}{l}
	\param{version-major},\param{version-minor}: CARD8
\end{tabular}\\
$\Rightarrow$\\
\begin{tabular}{l}
	version-major,version-minor: CARD8	
\end{tabular}
%end marker

This request must be executed before any other requests for this
extension.  If a client violates this rule, the results of all SYNC
requests that it issues are undefined.  The request takes the version
number of the extension that the client wishes to use and returns the
actual version number being implemented by the extension for this
client. The extension may return different version numbers to a client
depending of the version number supplied by that client. This request
should be executed only once for each client connection.

Given two different versions of the SYNC protocol, v1 and v2, v1 is
compatible with v2 if and only if $v1.version\_major = v2.version\_major$
and $v1.version\_minor \leq v2.version\_minor$.  Compatible means that the
functionality is fully supported in an identical fashion in the two
versions.

This document describes major version 3, minor version 0 of the SYNC
protocol.

% start marker
\requestdef{ListSystemCounters}

$\Rightarrow$\\
\begin{tabular}{l}
	system-counters: LISTofSYSTEMCOUNTER\\[5pt]
	Errors: \error{Alloc}
\end{tabular}
% end marker

This request returns a list of all the system counters that are available at
the time the request is executed, which includes the system counters that are
maintained by other extensions. The list returned by this request may change
as counters are created and destroyed by other extensions.

% start marker
\requestdef{CreateCounter}

\begin{tabular}{l}
	\param{id}: COUNTER\\
 	\param{initial-value}: INT64\\[5pt]
	Errors: \error{IDChoice},\error{Alloc}
\end{tabular}
% end marker

This request creates a counter and assigns the specified id to it.
The counter value is initialized to the specified initial-value
and there are no clients waiting on the counter.

% start marker
\requestdef{DestroyCounter}

\begin{tabular}{l}
	\param{counter}: COUNTER\\[5pt]
	Errors: \error{Counter},\error{Access}
\end{tabular}
% end marker

This request destroys the given counter and sets the counter fields
for all triggers that specify this counter to \enum{None}. All clients
waiting on the counter are released and a \event{CounterNotify} event with the
destroyed field set to TRUE is sent to each waiting client,
regardless of the event-threshold.  All alarms specifying the counter
become \enum{Inactive} and an \event{AlarmNotify} event with a state
field of \enum{Inactive} is generated. A counter is destroyed automatically
when the connection to the creating client is closed down if the close-down
mode is {\bf Destroy}. An \error{Access} error is generated if counter
is a system counter. A \error{Counter} error is generated if counter
does not name a valid counter.

% start marker
\requestdef{QueryCounter}

\begin{tabular}{l}
	\param{counter}: COUNTER\\
\end{tabular}\\
$\Rightarrow$\\
\begin{tabular}{l}
	value: INT64\\[5pt]
	Errors: \error{Counter}
\end{tabular}
% end marker

This request returns the current value of the given counter or a generates
\error{Counter} error if counter does not name a valid counter.

% start marker
\requestdef{Await}

\begin{tabular}{l}
	\param{wait-list}: LISTofWAITCONDITION\\[5pt]
	Errors: \error{Counter},\error{Alloc},\error{Value}
\end{tabular}
% end marker

When this request is executed, the triggers in the wait-list are
initialized using the wait-value and value-type fields, as
described in the definition of TRIGGER above. The processing of further
requests for the client is blocked until one or more of the triggers becomes
TRUE. This may happen immediately, as a result of the initialization, or at
some later time, as a result of a subsequent \request{SetCounter},
\request{ChangeCounter} or \request{DestroyCounter} request.

A \error{Value} error is generated if wait-list is empty.

When the client becomes unblocked, each trigger is checked to determine
whether a \event{CounterNotify} event should be generated. The difference
between the counter and the test value is calculated by
subtracting the test value from the value of the counter. If the
test-type is \enum{PositiveTransition} or \enum{PositiveComparison}, a \event{CounterNotify} event is generated if the
difference is at least event-threshold. If the test-type is
\enum{NegativeTransition} or \enum{NegativeComparison}, a
\event{CounterNotify} event is generated if the difference is at most
event-threshold. If the difference lies outside the range for an
INT64, an event is not generated.

This threshold check is made for each trigger in the list and a
\event{CounterNotify} event is generated for every trigger for which
the check succeeds. The check for \enum{CounterNotify} events is performed
even if one of the triggers is TRUE when the request is first executed. Note
that a \event{CounterNotify} event may be generated for a trigger that
is FALSE if there are multiple triggers in the request. A
\event{CounterNotify} event with the destroyed flag set to TRUE is
always generated if the counter for one of the triggers is destroyed.

% start marker
\requestdef{ChangeCounter}

\begin{tabular}{l}
	\param{counter}: COUNTER\\
	\param{amount}: INT64\\[5pt]
	Errors: \error{Counter},\error{Access},\error{Value}
\end{tabular}
% end marker

This request changes the given counter by adding amount to the current
counter value. If the change to this counter satisfies a trigger for which a
client is waiting, that client is unblocked and one or more
\event{CounterNotify} events may be generated. If the change to the counter
satisfies the trigger for an alarm, an \event{AlarmNotify} event is generated
and the alarm is updated.  An \error{Access} error is generated if
counter is a system counter. A \error{Counter} error is generated if
counter does not name a valid counter. If the resulting value for the
counter would be outside the range for an INT64, a \error{Value} error is
generated and the counter is not changed.

It should be noted that all the clients whose triggers are satisfied by
this change are unblocked, so this request cannot be used to implement mutual
exclusion.

% start marker
\requestdef{SetCounter}

\begin{tabular}{l}
	\param{counter}: COUNTER\\
	\param{value}: INT64\\[5pt]
	Errors: \error{Counter},\error{Access}
\end{tabular}
% end marker

This request sets the value of the given counter to value. The effect
is equivalent to executing the appropriate \request{ChangeCounter} request to
change the counter value to value. An \error{Access} error is
generated if counter names a system counter. A \error{Counter} error
is generated if counter does not name a valid counter.

% start marker
\requestdef{CreateAlarm}

\begin{tabular}{l}
	\param{id}: ALARM\\
 	\param{values-mask}: CARD32\\
        \param{values-list}: LISTofVALUE\\[5pt]
 	Errors: \error{IDChoice},\error{Counter},\error{Match},\error{Value},\error{Alloc}
\end{tabular}
% end marker

This request creates an alarm and assigns the identifier id to it. The
values-mask and values-list specify the attributes that are
to be explicitly initialized. The attributes for an Alarm and their defaults
are:

\begin{center}
\begin{tabular}{l|l|ll}
Attribute	& Type		& Default \\
\hline
trigger		& TRIGGER	& counter	& \enum{None}\\
		&		& value-type	& \enum{Absolute}\\
		&		& value	& 0\\
		&		& test-type	& \enum{PositiveComparison}\\
delta		& INT64		& 1 \\
events		& BOOL		& TRUE
\end{tabular}
\end{center}

The trigger is initialized as described in the definition of TRIGGER,
with an error being generated if necessary.

If the counter is \enum{None}, the state of the alarm is set to
\enum{Inactive}, else it is set to \enum{Active}.

Whenever the trigger becomes TRUE, either as a result of this request
or as the result of a \request{SetCounter}, \request{ChangeCounter},
\request{DestroyCounter}, or \request{ChangeAlarm} request, an
\event{AlarmNotify} event is generated and the alarm is updated. The alarm is
updated by repeatedly adding delta to the value of the
trigger and reinitializing it until it becomes FALSE. If this update
would cause value to fall outside the range for an INT64, or if the
counter value is \enum{None}, or if the
delta is 0 and test-type is \enum{PositiveComparison} or
\enum{NegativeComparison}, no change is made to value and the alarm
state is changed to \enum{Inactive} before the event is generated. No further
events are generated by an \enum{Inactive} alarm until a \request{ChangeAlarm}
or \request{DestroyAlarm} request is executed.

If the test-type is \enum{PositiveComparison} or
\enum{PositiveTransition} and delta is less than zero, or
if the test-type is \enum{NegativeComparison} or
\enum{NegativeTransition} and delta is greater than zero,
a \error{Match} error is generated.

The events value enables or disables delivery of \event{AlarmNotify}
events to the requesting client.  The alarm keeps a separate event flag for
each client so that other clients may select to receive events from this
alarm.

An \event{AlarmNotify} event is always generated at some time after the
execution of a \request{CreateAlarm} request. This will happen immediately if
the trigger is TRUE, or it will happen later when the
trigger becomes TRUE or the Alarm is destroyed.

% start marker
\requestdef{ChangeAlarm}

\begin{tabular}{l}
	\param{id}: ALARM\\
	\param{values-mask}: CARD32\\
	\param{values-list}: LISTofVALUE\\[5pt]
	Errors: \error{Alarm},\error{Counter},\error{Value},\error{Match}
\end{tabular}
% end marker

This request changes the parameters of an Alarm. All of the parameters
specified for the \request{CreateAlarm} request may be changed using this
request. The trigger is reinitialized and an \event{AlarmNotify}
event is generated if appropriate, as explained in the description of the
\request{CreateAlarm} request.

Changes to the events flag affect the event delivery to the requesting
client only and may be used by a client to select or deselect event delivery
from an alarm created by another client.

The order in which attributes are verified and altered is
server-dependent.  If an error is generated, a subset of the
attributes may have been altered.

% start marker
\requestdef{DestroyAlarm}

\begin{tabular}{l}
	\param{alarm}: ALARM\\[5pt] 	Errors: \error{Alarm}
\end{tabular}
% end marker

This request destroys an alarm. An alarm is automatically destroyed
when the creating client is closed down if the close-down mode is {\bf
Destroy}. When an alarm is destroyed, an \event{AlarmNotify} event is
generated with a state value of \enum{Destroyed}.

% start marker
\requestdef{QueryAlarm}

\begin{tabular}{l}
	\param{alarm}: ALARM\\
\end{tabular}\\
$\Rightarrow$\\
\begin{tabular}{l}
	trigger: TRIGGER\\
	delta: INT64\\
	events: ALARMEVENTMASK\\
	state: ALARMSTATE\\[5pt]
	Errors: \error{Alarm}
\end{tabular}
% end marker

This request retrieves the current parameters for an Alarm.

% start marker
\requestdef{SetPriority}

\begin{tabular}{l}
	\param{client-resource}: XID\\
	\param{priority}: INT32\\[5pt]
	Errors: \error{Match}
\end{tabular}
% end marker

This request changes the scheduling priority of the client that created
client-resource. If client-resource is \enum{None}, then the
priority for the client making the request is changed. A \error{Match} error
is generated if client-resource is not \enum{None} and does not name
an existing resource in the server.  For any two priority values,
{\tt A} and {\tt B}, {\tt A} is higher priority if and only if {\tt A} is
greater than {\tt B}.

The priority of a client is set to 0 when the initial client connection is
made.

The effect of different client priorities depends on the particular
implementation of the extension, and in some cases it may have no effect at
all. However, the intention is that higher priority clients will have their
requests executed before those of lower priority clients.

For most animation applications, it is desirable that animation clients be
given priority over nonrealtime clients. This improves the smoothness of the
animation on a loaded server. Because a server is free to implement very strict
priorities, processing requests for the highest priority client to the
exclusion of all others, it is important that a client that may potentially
monopolize the whole server, such as an animation that produces continuous
output as fast as it can with no rate control, is run at low rather than high
priority.

% start marker
\requestdef{GetPriority}

\begin{tabular}{l}
	\param{client-resource}: XID\\
\end{tabular}\\
$\Rightarrow$\\
\begin{tabular}{l}
	priority: INT32\\[5pt]
	Errors: \error{Match}
\end{tabular}
% end marker

This request returns the scheduling priority of the client that created
client-resource. If client-resource is \enum{None}, then the
priority for the client making the request is returned. A \error{Match} error
is generated if client-resource is not \enum{None} and does not name
an existing resource in the server.

\end{description}

\subsection{Events}

\begin{description}

% start marker
\eventdef{CounterNotify}

\begin{tabular}{l}
	\param{counter}: COUNTER \\
 	\param{wait-value}: INT64 \\
	\param{counter-value}: INT64 \\
 	\param{time}: TIME \\
	\param{count}: CARD16 \\
	\param{destroyed}: BOOL
\end{tabular}
% end marker

\event{CounterNotify} events may be generated when a client becomes unblocked
after an \request{Await} request has been processed.
The wait-value is the value being waited for, and
counter-value is the actual value of the counter at the time
the event was generated. The
destroyed flag is TRUE if this request was generated as the
result of the destruction of the counter and FALSE otherwise.
The time is the server time at which the event was generated.

When a client is unblocked, all the \event{CounterNotify} events for the
\request{Await} request are generated contiguously. If
count is 0, there are no more events to follow for this request. If
count is $n$, there are at least $n$ more events to follow.

% start marker
\eventdef{AlarmNotify}

\begin{tabular}{l}
	\param{alarm}: ALARM \\
 	\param{counter-value}: INT64 \\
	\param{alarm-value}: INT64 \\
 	\param{state}: ALARMSTATE \\
 	\param{time}: TIME
\end{tabular}
% end marker

An \event{AlarmNotify} event is generated when an alarm is triggered.
alarm-value is the test value of the trigger in the alarm when it was
triggered, counter-value is the value of the counter that triggered
the alarm, and time is the server time at which the event was
generated. The state is the new state of the alarm. If state is
\enum{Inactive}, no more events will be generated by this alarm until a
\request{ChangeAlarm} request is executed, the alarm is destroyed, or the
counter for the alarm is destroyed.

\end{description}

\section{Encoding}

Please refer to the X11 Protocol Encoding document as this section uses
syntactic conventions established there and references types defined there.

The name of this extension is ``SYNC''.

\subsection{New Types}

The following new types are used by the extension.

\begin{tabbing}
\tabstopsC
ALARM: CARD32\\
ALARMSTATE:\\
\tabstopsB
	\> 0	\> Active \\
	\> 1	\> Inactive \\
	\> 2	\> Destroyed\\
\tabstopsC
COUNTER: CARD32\\
INT64: 64-bit signed integer\\
SYSTEMCOUNTER:\\
	\> 4	\> COUNTER	\> counter \\
	\> 8	\> INT64	\> resolution\\
	\> 2	\> n		\> length of name in bytes\\
	\> n	\> STRING8	\> name \\
	\> p	\> 		\> pad,p=pad(n+2)\\
TESTTYPE:\\
\tabstopsB
	\> 0	\> PositiveTransition \\
	\> 1	\> NegativeTransition \\
	\> 2	\> PositiveComparison \\
	\> 3	\> NegativeComparison \\
\tabstopsC
TRIGGER:\\
	\> 4	\> COUNTER	\> counter \\
	\> 4	\> VALUETYPE	\> wait-type \\	
	\> 8	\> INT64	\> wait-value \\
	\> 4	\> TESTTYPE	\> test-type \\
VALUETYPE:\\
\tabstopsB
	\> 0	\> Absolute \\
	\> 1	\> Relative \\
\tabstopsC
WAITCONDITION:\\
	\> 20	\> TRIGGER	\> trigger \\
	\> 8	\> INT64	\> event threshold\\
\end{tabbing}

An INT64 is encoded in 8 bytes with the most significant 4 bytes
first followed by the least significant 4 bytes.  Within these
4-byte groups, the byte ordering determined during connection setup
is used.

\subsection{Errors}

\begin{tabbing}
\tabstopsC
{\bf Counter}\\
	\> 1	\> 0		\> Error \\
	\> 1	\> Base + 0	\> code \\
	\> 2	\> CARD16	\> sequence number \\
	\> 4	\> CARD32	\> bad counter \\
	\> 2	\> CARD16	\> minor opcode \\
	\> 1	\> CARD8	\> major opcode \\
	\> 21	\>		\> unused \\
{\bf Alarm}\\
	\> 1	\> 0		\> Error \\
	\> 1	\> Base + 1	\> code \\
	\> 2	\> CARD16	\> sequence number \\
	\> 4	\> CARD32	\> bad alarm \\
	\> 2	\> CARD16	\> minor opcode \\
	\> 1	\> CARD8	\> major opcode \\
	\> 21	\>		\> unused \\
\end{tabbing}

\subsection{Requests}

\renewcommand{\thefootnote}{\fnsymbol{footnote}}
\setcounter{footnote}{1}
\setlength{\topsep}{0pt}	%vertical space before and after tabbing
\begin{tabbing}
\tabstopsC
{\bf Initialize}\\
	\> 1	\> CARD8	\> major opcode \\
	\> 1	\> 0		\> minor opcode \\
	\> 2	\> 2		\> request length \\
	\> 1	\> CARD8	\> major version \\
	\> 1	\> CARD8	\> minor version \\
	\> 2	\> 		\> unused \\
$\Rightarrow$\\
	\> 1	\> 1		\> Reply \\
	\> 1	\>		\> unused \\
	\> 2	\> CARD16	\> sequence number \\
	\> 4	\> 0		\> reply length \\
	\> 1	\> CARD8	\> major version \\
	\> 1	\> CARD8	\> minor version \\
	\> 2	\>		\> unused \\
	\> 20	\>		\> unused \\
\\
{\bf ListSystemCounters}\\
	\> 1	\> CARD8	\> major opcode \\
	\> 1	\> 1		\> minor opcode \\
	\> 2	\> 1		\> request length \\
$\Rightarrow$\\
	\> 1	\> 1		\> Reply \\
	\> 1	\>		\> unused \\
	\> 2	\> CARD16	\> sequence number \\
	\> 4	\> {\it variable} \> reply length \\
	\> 4	\> INT32	\> list length \\
	\> 20	\>		\> unused \\
	\> 4n   \> list of SYSTEMCOUNTER \> system counters \\
\\
{\bf CreateCounter}\\
	\> 1	\> CARD8	\> major opcode \\
	\> 1	\> 2		\> minor opcode \\
	\> 2	\> 4		\> request length \\
	\> 4	\> COUNTER	\> id\\
	\> 8	\> INT64	\> initial value\\
\\
{\bf DestroyCounter}\\
	\> 1	\> CARD8	\> major opcode \\
	\> 1	\> 6		\> minor opcode\footnotemark[1] \\
	\> 2	\> 2		\> request length \\
	\> 4	\> COUNTER	\> counter
\end{tabbing}
\footnotetext{A previous version of this document gave an incorrect
minor opcode.}
\begin{tabbing}
\tabstopsC
{\bf QueryCounter}\\
	\> 1	\> CARD8	\> major opcode \\
	\> 1	\> 5		\> minor opcode\footnotemark[1] \\
	\> 2	\> 2		\> request length \\
	\> 4	\> COUNTER	\> counter \\
$\Rightarrow$\\
	\> 1	\> 1		\> Reply \\
	\> 1	\>		\> unused \\
	\> 2	\> CARD16	\> sequence number \\
	\> 4	\> 0		\> reply length \\
	\> 8	\> INT64	\> counter value \\
	\> 16	\>		\> unused\\
\\
{\bf Await}\\
	\> 1	\> CARD8	\> major opcode \\
	\> 1	\> 7		\> minor opcode\footnotemark[1] \\
	\> 2	\> 1 + 7*n	\> request length \\
	\> 28n	\> LISTofWAITCONDITION \> wait conditions
\end{tabbing}
\footnotetext{A previous version of this document gave an incorrect
minor opcode.}
\setlength{\topsep}{0pt}	%vertical space before and after tabbing
\begin{tabbing}
\tabstopsC
{\bf ChangeCounter}\\
	\> 1	\> CARD8	\> major opcode \\
	\> 1	\> 4		\> minor opcode\footnotemark[1] \\
	\> 2	\> 4		\> request length \\
	\> 4	\> COUNTER	\> counter \\
	\> 8	\> INT64	\> amount \\
\\
{\bf SetCounter}\\
	\> 1	\> CARD8	\> major opcode \\
	\> 1	\> 3		\> minor opcode\footnotemark[1] \\
	\> 2	\> 4		\> request length \\
	\> 4	\> COUNTER	\> counter \\
	\> 8	\> INT64	\> value \\
\\
{\bf CreateAlarm}\\
	\> 1	\> CARD8	\> major opcode \\
	\> 1	\> 8		\> minor opcode \\
	\> 2	\> 3+n		\> request length \\
	\> 4	\> ALARM	\> id \\
	\> 4	\> BITMASK	\> values mask\\
\tabstopsB
	\>	\> \#x00000001	\> counter \\
	\>	\> \#x00000002	\> value-type \\
	\>	\> \#x00000004	\> value \\
	\>	\> \#x00000008	\> test-type \\
	\>	\> \#x00000010	\> delta \\
	\>	\>  \#x00000020	\> events \\
\tabstopsC
	\> 4n	\> LISTofVALUE	\> values\\
\tabstopsB
VALUES\\
	\> 4	\> COUNTER	\> counter\\
	\> 4	\> VALUETYPE	\> value-type \\
	\> 8	\> INT64	\> value \\
	\> 4	\> TESTTYPE	\> test-type \\
	\> 8	\> INT64	\> delta \\
	\> 4	\> BOOL		\> events\\
\tabstopsC
\\
{\bf ChangeAlarm}\\
	\> 1	\> CARD8	\> major opcode \\
	\> 1	\> 9		\> minor opcode \\
	\> 2	\> 3+n		\> request length \\
	\> 4	\> ALARM	\> id \\
	\> 4	\> BITMASK	\> values mask \\
	\> 	\> encodings as for {\bf CreateAlarm}\\
	\> 4n	\> LISTofVALUE	\> values\\
	\>	\> encodings as for {\bf CreateAlarm}\\
\\
{\bf DestroyAlarm}\\
	\> 1	\> CARD8	\> major opcode \\
	\> 1	\> 11		\> minor opcode\footnotemark[1] \\
	\> 2	\> 2		\> request length \\
	\> 4	\> ALARM	\> alarm
\end{tabbing}
\footnotetext{A previous version of this document gave an incorrect
minor opcode.}
\begin{tabbing}
\tabstopsC
{\bf QueryAlarm}\\
	\> 1	\> CARD8	\> major opcode \\
	\> 1	\> 10		\> minor opcode\footnotemark[1] \\
	\> 2	\> 2		\> request length \\
	\> 4	\> ALARM	\> alarm \\
$\Rightarrow$\\
	\> 1	\> 1		\> Reply \\
	\> 1	\>		\> unused \\
	\> 2	\> CARD16	\> sequence number \\
	\> 4	\> 2		\> reply length \\
	\> 20	\> TRIGGER	\> trigger \\
	\> 8	\> INT64	\> delta \\
	\> 1	\> BOOL		\> events \\
	\> 1	\> ALARMSTATE	\> state \\
	\> 2	\>		\> unused \\
\\
{\bf SetPriority}\\
	\> 1	\> CARD8	\> major opcode \\
	\> 1	\> 12		\> minor opcode \\
	\> 2	\> 3		\> request length \\
	\> 4	\> CARD32	\> id \\
	\> 4	\> INT32	\> priority \\
\\
{\bf GetPriority}\\
	\> 1	\> CARD8	\> major opcode \\
	\> 1	\> 13		\> minor opcode \\
	\> 2	\> 1		\> request length \\
	\> 4	\> CARD32	\> id \\
$\Rightarrow$\\
	\> 1	\> 1		\> Reply \\
	\> 1	\>		\> unused \\
	\> 2	\> CARD16	\> sequence number \\
	\> 4	\> 0		\> reply length \\
	\> 4	\> INT32	\> priority \\
	\> 20	\>		\> unused\\
\end{tabbing}

\subsection{Events}

\begin{tabbing}
\tabstopsC
{\bf CounterNotify}\\
	\> 1	\> Base + 0	\> code \\
	\> 1	\> 0		\> kind \\
	\> 2	\> CARD16	\> sequence number \\
	\> 4	\> COUNTER	\> counter \\
	\> 8	\> INT64	\> wait value \\
	\> 8	\> INT64	\> counter value \\
	\> 4	\> TIME		\> timestamp \\
	\> 2	\> CARD16	\> count \\
	\> 1	\> BOOL		\> destroyed \\
	\> 1	\> 		\> unused \\
\\
{\bf AlarmNotify}\\
	\> 1	\> Base + 1	\> code \\
	\> 1	\> 1		\> kind \\
	\> 2	\> CARD16	\> sequence number \\
	\> 4	\> ALARM	\> alarm \\
	\> 8	\> INT64	\> counter value \\
	\> 8	\> INT64	\> alarm value \\
	\> 4	\> TIME		\> timestamp \\
	\> 1	\> ALARMSTATE	\> state \\
	\> 3	\>		\> unused\\
\end{tabbing}
\end{document}
